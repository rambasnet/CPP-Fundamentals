\documentclass[11pt]{article}

    \usepackage[breakable]{tcolorbox}
    \usepackage{parskip} % Stop auto-indenting (to mimic markdown behaviour)
    
    \usepackage{iftex}
    \ifPDFTeX
    	\usepackage[T1]{fontenc}
    	\usepackage{mathpazo}
    \else
    	\usepackage{fontspec}
    \fi

    % Basic figure setup, for now with no caption control since it's done
    % automatically by Pandoc (which extracts ![](path) syntax from Markdown).
    \usepackage{graphicx}
    % Maintain compatibility with old templates. Remove in nbconvert 6.0
    \let\Oldincludegraphics\includegraphics
    % Ensure that by default, figures have no caption (until we provide a
    % proper Figure object with a Caption API and a way to capture that
    % in the conversion process - todo).
    \usepackage{caption}
    \DeclareCaptionFormat{nocaption}{}
    \captionsetup{format=nocaption,aboveskip=0pt,belowskip=0pt}

    \usepackage{float}
    \floatplacement{figure}{H} % forces figures to be placed at the correct location
    \usepackage{xcolor} % Allow colors to be defined
    \usepackage{enumerate} % Needed for markdown enumerations to work
    \usepackage{geometry} % Used to adjust the document margins
    \usepackage{amsmath} % Equations
    \usepackage{amssymb} % Equations
    \usepackage{textcomp} % defines textquotesingle
    % Hack from http://tex.stackexchange.com/a/47451/13684:
    \AtBeginDocument{%
        \def\PYZsq{\textquotesingle}% Upright quotes in Pygmentized code
    }
    \usepackage{upquote} % Upright quotes for verbatim code
    \usepackage{eurosym} % defines \euro
    \usepackage[mathletters]{ucs} % Extended unicode (utf-8) support
    \usepackage{fancyvrb} % verbatim replacement that allows latex
    \usepackage{grffile} % extends the file name processing of package graphics 
                         % to support a larger range
    \makeatletter % fix for old versions of grffile with XeLaTeX
    \@ifpackagelater{grffile}{2019/11/01}
    {
      % Do nothing on new versions
    }
    {
      \def\Gread@@xetex#1{%
        \IfFileExists{"\Gin@base".bb}%
        {\Gread@eps{\Gin@base.bb}}%
        {\Gread@@xetex@aux#1}%
      }
    }
    \makeatother
    \usepackage[Export]{adjustbox} % Used to constrain images to a maximum size
    \adjustboxset{max size={0.9\linewidth}{0.9\paperheight}}

    % The hyperref package gives us a pdf with properly built
    % internal navigation ('pdf bookmarks' for the table of contents,
    % internal cross-reference links, web links for URLs, etc.)
    \usepackage{hyperref}
    % The default LaTeX title has an obnoxious amount of whitespace. By default,
    % titling removes some of it. It also provides customization options.
    \usepackage{titling}
    \usepackage{longtable} % longtable support required by pandoc >1.10
    \usepackage{booktabs}  % table support for pandoc > 1.12.2
    \usepackage[inline]{enumitem} % IRkernel/repr support (it uses the enumerate* environment)
    \usepackage[normalem]{ulem} % ulem is needed to support strikethroughs (\sout)
                                % normalem makes italics be italics, not underlines
    \usepackage{mathrsfs}
    

    
    % Colors for the hyperref package
    \definecolor{urlcolor}{rgb}{0,.145,.698}
    \definecolor{linkcolor}{rgb}{.71,0.21,0.01}
    \definecolor{citecolor}{rgb}{.12,.54,.11}

    % ANSI colors
    \definecolor{ansi-black}{HTML}{3E424D}
    \definecolor{ansi-black-intense}{HTML}{282C36}
    \definecolor{ansi-red}{HTML}{E75C58}
    \definecolor{ansi-red-intense}{HTML}{B22B31}
    \definecolor{ansi-green}{HTML}{00A250}
    \definecolor{ansi-green-intense}{HTML}{007427}
    \definecolor{ansi-yellow}{HTML}{DDB62B}
    \definecolor{ansi-yellow-intense}{HTML}{B27D12}
    \definecolor{ansi-blue}{HTML}{208FFB}
    \definecolor{ansi-blue-intense}{HTML}{0065CA}
    \definecolor{ansi-magenta}{HTML}{D160C4}
    \definecolor{ansi-magenta-intense}{HTML}{A03196}
    \definecolor{ansi-cyan}{HTML}{60C6C8}
    \definecolor{ansi-cyan-intense}{HTML}{258F8F}
    \definecolor{ansi-white}{HTML}{C5C1B4}
    \definecolor{ansi-white-intense}{HTML}{A1A6B2}
    \definecolor{ansi-default-inverse-fg}{HTML}{FFFFFF}
    \definecolor{ansi-default-inverse-bg}{HTML}{000000}

    % common color for the border for error outputs.
    \definecolor{outerrorbackground}{HTML}{FFDFDF}

    % commands and environments needed by pandoc snippets
    % extracted from the output of `pandoc -s`
    \providecommand{\tightlist}{%
      \setlength{\itemsep}{0pt}\setlength{\parskip}{0pt}}
    \DefineVerbatimEnvironment{Highlighting}{Verbatim}{commandchars=\\\{\}}
    % Add ',fontsize=\small' for more characters per line
    \newenvironment{Shaded}{}{}
    \newcommand{\KeywordTok}[1]{\textcolor[rgb]{0.00,0.44,0.13}{\textbf{{#1}}}}
    \newcommand{\DataTypeTok}[1]{\textcolor[rgb]{0.56,0.13,0.00}{{#1}}}
    \newcommand{\DecValTok}[1]{\textcolor[rgb]{0.25,0.63,0.44}{{#1}}}
    \newcommand{\BaseNTok}[1]{\textcolor[rgb]{0.25,0.63,0.44}{{#1}}}
    \newcommand{\FloatTok}[1]{\textcolor[rgb]{0.25,0.63,0.44}{{#1}}}
    \newcommand{\CharTok}[1]{\textcolor[rgb]{0.25,0.44,0.63}{{#1}}}
    \newcommand{\StringTok}[1]{\textcolor[rgb]{0.25,0.44,0.63}{{#1}}}
    \newcommand{\CommentTok}[1]{\textcolor[rgb]{0.38,0.63,0.69}{\textit{{#1}}}}
    \newcommand{\OtherTok}[1]{\textcolor[rgb]{0.00,0.44,0.13}{{#1}}}
    \newcommand{\AlertTok}[1]{\textcolor[rgb]{1.00,0.00,0.00}{\textbf{{#1}}}}
    \newcommand{\FunctionTok}[1]{\textcolor[rgb]{0.02,0.16,0.49}{{#1}}}
    \newcommand{\RegionMarkerTok}[1]{{#1}}
    \newcommand{\ErrorTok}[1]{\textcolor[rgb]{1.00,0.00,0.00}{\textbf{{#1}}}}
    \newcommand{\NormalTok}[1]{{#1}}
    
    % Additional commands for more recent versions of Pandoc
    \newcommand{\ConstantTok}[1]{\textcolor[rgb]{0.53,0.00,0.00}{{#1}}}
    \newcommand{\SpecialCharTok}[1]{\textcolor[rgb]{0.25,0.44,0.63}{{#1}}}
    \newcommand{\VerbatimStringTok}[1]{\textcolor[rgb]{0.25,0.44,0.63}{{#1}}}
    \newcommand{\SpecialStringTok}[1]{\textcolor[rgb]{0.73,0.40,0.53}{{#1}}}
    \newcommand{\ImportTok}[1]{{#1}}
    \newcommand{\DocumentationTok}[1]{\textcolor[rgb]{0.73,0.13,0.13}{\textit{{#1}}}}
    \newcommand{\AnnotationTok}[1]{\textcolor[rgb]{0.38,0.63,0.69}{\textbf{\textit{{#1}}}}}
    \newcommand{\CommentVarTok}[1]{\textcolor[rgb]{0.38,0.63,0.69}{\textbf{\textit{{#1}}}}}
    \newcommand{\VariableTok}[1]{\textcolor[rgb]{0.10,0.09,0.49}{{#1}}}
    \newcommand{\ControlFlowTok}[1]{\textcolor[rgb]{0.00,0.44,0.13}{\textbf{{#1}}}}
    \newcommand{\OperatorTok}[1]{\textcolor[rgb]{0.40,0.40,0.40}{{#1}}}
    \newcommand{\BuiltInTok}[1]{{#1}}
    \newcommand{\ExtensionTok}[1]{{#1}}
    \newcommand{\PreprocessorTok}[1]{\textcolor[rgb]{0.74,0.48,0.00}{{#1}}}
    \newcommand{\AttributeTok}[1]{\textcolor[rgb]{0.49,0.56,0.16}{{#1}}}
    \newcommand{\InformationTok}[1]{\textcolor[rgb]{0.38,0.63,0.69}{\textbf{\textit{{#1}}}}}
    \newcommand{\WarningTok}[1]{\textcolor[rgb]{0.38,0.63,0.69}{\textbf{\textit{{#1}}}}}
    
    
    % Define a nice break command that doesn't care if a line doesn't already
    % exist.
    \def\br{\hspace*{\fill} \\* }
    % Math Jax compatibility definitions
    \def\gt{>}
    \def\lt{<}
    \let\Oldtex\TeX
    \let\Oldlatex\LaTeX
    \renewcommand{\TeX}{\textrm{\Oldtex}}
    \renewcommand{\LaTeX}{\textrm{\Oldlatex}}
    % Document parameters
    % Document title
    \title{StdInputOutput}
    
    
    
    
    
% Pygments definitions
\makeatletter
\def\PY@reset{\let\PY@it=\relax \let\PY@bf=\relax%
    \let\PY@ul=\relax \let\PY@tc=\relax%
    \let\PY@bc=\relax \let\PY@ff=\relax}
\def\PY@tok#1{\csname PY@tok@#1\endcsname}
\def\PY@toks#1+{\ifx\relax#1\empty\else%
    \PY@tok{#1}\expandafter\PY@toks\fi}
\def\PY@do#1{\PY@bc{\PY@tc{\PY@ul{%
    \PY@it{\PY@bf{\PY@ff{#1}}}}}}}
\def\PY#1#2{\PY@reset\PY@toks#1+\relax+\PY@do{#2}}

\@namedef{PY@tok@w}{\def\PY@tc##1{\textcolor[rgb]{0.73,0.73,0.73}{##1}}}
\@namedef{PY@tok@c}{\let\PY@it=\textit\def\PY@tc##1{\textcolor[rgb]{0.25,0.50,0.50}{##1}}}
\@namedef{PY@tok@cp}{\def\PY@tc##1{\textcolor[rgb]{0.74,0.48,0.00}{##1}}}
\@namedef{PY@tok@k}{\let\PY@bf=\textbf\def\PY@tc##1{\textcolor[rgb]{0.00,0.50,0.00}{##1}}}
\@namedef{PY@tok@kp}{\def\PY@tc##1{\textcolor[rgb]{0.00,0.50,0.00}{##1}}}
\@namedef{PY@tok@kt}{\def\PY@tc##1{\textcolor[rgb]{0.69,0.00,0.25}{##1}}}
\@namedef{PY@tok@o}{\def\PY@tc##1{\textcolor[rgb]{0.40,0.40,0.40}{##1}}}
\@namedef{PY@tok@ow}{\let\PY@bf=\textbf\def\PY@tc##1{\textcolor[rgb]{0.67,0.13,1.00}{##1}}}
\@namedef{PY@tok@nb}{\def\PY@tc##1{\textcolor[rgb]{0.00,0.50,0.00}{##1}}}
\@namedef{PY@tok@nf}{\def\PY@tc##1{\textcolor[rgb]{0.00,0.00,1.00}{##1}}}
\@namedef{PY@tok@nc}{\let\PY@bf=\textbf\def\PY@tc##1{\textcolor[rgb]{0.00,0.00,1.00}{##1}}}
\@namedef{PY@tok@nn}{\let\PY@bf=\textbf\def\PY@tc##1{\textcolor[rgb]{0.00,0.00,1.00}{##1}}}
\@namedef{PY@tok@ne}{\let\PY@bf=\textbf\def\PY@tc##1{\textcolor[rgb]{0.82,0.25,0.23}{##1}}}
\@namedef{PY@tok@nv}{\def\PY@tc##1{\textcolor[rgb]{0.10,0.09,0.49}{##1}}}
\@namedef{PY@tok@no}{\def\PY@tc##1{\textcolor[rgb]{0.53,0.00,0.00}{##1}}}
\@namedef{PY@tok@nl}{\def\PY@tc##1{\textcolor[rgb]{0.63,0.63,0.00}{##1}}}
\@namedef{PY@tok@ni}{\let\PY@bf=\textbf\def\PY@tc##1{\textcolor[rgb]{0.60,0.60,0.60}{##1}}}
\@namedef{PY@tok@na}{\def\PY@tc##1{\textcolor[rgb]{0.49,0.56,0.16}{##1}}}
\@namedef{PY@tok@nt}{\let\PY@bf=\textbf\def\PY@tc##1{\textcolor[rgb]{0.00,0.50,0.00}{##1}}}
\@namedef{PY@tok@nd}{\def\PY@tc##1{\textcolor[rgb]{0.67,0.13,1.00}{##1}}}
\@namedef{PY@tok@s}{\def\PY@tc##1{\textcolor[rgb]{0.73,0.13,0.13}{##1}}}
\@namedef{PY@tok@sd}{\let\PY@it=\textit\def\PY@tc##1{\textcolor[rgb]{0.73,0.13,0.13}{##1}}}
\@namedef{PY@tok@si}{\let\PY@bf=\textbf\def\PY@tc##1{\textcolor[rgb]{0.73,0.40,0.53}{##1}}}
\@namedef{PY@tok@se}{\let\PY@bf=\textbf\def\PY@tc##1{\textcolor[rgb]{0.73,0.40,0.13}{##1}}}
\@namedef{PY@tok@sr}{\def\PY@tc##1{\textcolor[rgb]{0.73,0.40,0.53}{##1}}}
\@namedef{PY@tok@ss}{\def\PY@tc##1{\textcolor[rgb]{0.10,0.09,0.49}{##1}}}
\@namedef{PY@tok@sx}{\def\PY@tc##1{\textcolor[rgb]{0.00,0.50,0.00}{##1}}}
\@namedef{PY@tok@m}{\def\PY@tc##1{\textcolor[rgb]{0.40,0.40,0.40}{##1}}}
\@namedef{PY@tok@gh}{\let\PY@bf=\textbf\def\PY@tc##1{\textcolor[rgb]{0.00,0.00,0.50}{##1}}}
\@namedef{PY@tok@gu}{\let\PY@bf=\textbf\def\PY@tc##1{\textcolor[rgb]{0.50,0.00,0.50}{##1}}}
\@namedef{PY@tok@gd}{\def\PY@tc##1{\textcolor[rgb]{0.63,0.00,0.00}{##1}}}
\@namedef{PY@tok@gi}{\def\PY@tc##1{\textcolor[rgb]{0.00,0.63,0.00}{##1}}}
\@namedef{PY@tok@gr}{\def\PY@tc##1{\textcolor[rgb]{1.00,0.00,0.00}{##1}}}
\@namedef{PY@tok@ge}{\let\PY@it=\textit}
\@namedef{PY@tok@gs}{\let\PY@bf=\textbf}
\@namedef{PY@tok@gp}{\let\PY@bf=\textbf\def\PY@tc##1{\textcolor[rgb]{0.00,0.00,0.50}{##1}}}
\@namedef{PY@tok@go}{\def\PY@tc##1{\textcolor[rgb]{0.53,0.53,0.53}{##1}}}
\@namedef{PY@tok@gt}{\def\PY@tc##1{\textcolor[rgb]{0.00,0.27,0.87}{##1}}}
\@namedef{PY@tok@err}{\def\PY@bc##1{{\setlength{\fboxsep}{\string -\fboxrule}\fcolorbox[rgb]{1.00,0.00,0.00}{1,1,1}{\strut ##1}}}}
\@namedef{PY@tok@kc}{\let\PY@bf=\textbf\def\PY@tc##1{\textcolor[rgb]{0.00,0.50,0.00}{##1}}}
\@namedef{PY@tok@kd}{\let\PY@bf=\textbf\def\PY@tc##1{\textcolor[rgb]{0.00,0.50,0.00}{##1}}}
\@namedef{PY@tok@kn}{\let\PY@bf=\textbf\def\PY@tc##1{\textcolor[rgb]{0.00,0.50,0.00}{##1}}}
\@namedef{PY@tok@kr}{\let\PY@bf=\textbf\def\PY@tc##1{\textcolor[rgb]{0.00,0.50,0.00}{##1}}}
\@namedef{PY@tok@bp}{\def\PY@tc##1{\textcolor[rgb]{0.00,0.50,0.00}{##1}}}
\@namedef{PY@tok@fm}{\def\PY@tc##1{\textcolor[rgb]{0.00,0.00,1.00}{##1}}}
\@namedef{PY@tok@vc}{\def\PY@tc##1{\textcolor[rgb]{0.10,0.09,0.49}{##1}}}
\@namedef{PY@tok@vg}{\def\PY@tc##1{\textcolor[rgb]{0.10,0.09,0.49}{##1}}}
\@namedef{PY@tok@vi}{\def\PY@tc##1{\textcolor[rgb]{0.10,0.09,0.49}{##1}}}
\@namedef{PY@tok@vm}{\def\PY@tc##1{\textcolor[rgb]{0.10,0.09,0.49}{##1}}}
\@namedef{PY@tok@sa}{\def\PY@tc##1{\textcolor[rgb]{0.73,0.13,0.13}{##1}}}
\@namedef{PY@tok@sb}{\def\PY@tc##1{\textcolor[rgb]{0.73,0.13,0.13}{##1}}}
\@namedef{PY@tok@sc}{\def\PY@tc##1{\textcolor[rgb]{0.73,0.13,0.13}{##1}}}
\@namedef{PY@tok@dl}{\def\PY@tc##1{\textcolor[rgb]{0.73,0.13,0.13}{##1}}}
\@namedef{PY@tok@s2}{\def\PY@tc##1{\textcolor[rgb]{0.73,0.13,0.13}{##1}}}
\@namedef{PY@tok@sh}{\def\PY@tc##1{\textcolor[rgb]{0.73,0.13,0.13}{##1}}}
\@namedef{PY@tok@s1}{\def\PY@tc##1{\textcolor[rgb]{0.73,0.13,0.13}{##1}}}
\@namedef{PY@tok@mb}{\def\PY@tc##1{\textcolor[rgb]{0.40,0.40,0.40}{##1}}}
\@namedef{PY@tok@mf}{\def\PY@tc##1{\textcolor[rgb]{0.40,0.40,0.40}{##1}}}
\@namedef{PY@tok@mh}{\def\PY@tc##1{\textcolor[rgb]{0.40,0.40,0.40}{##1}}}
\@namedef{PY@tok@mi}{\def\PY@tc##1{\textcolor[rgb]{0.40,0.40,0.40}{##1}}}
\@namedef{PY@tok@il}{\def\PY@tc##1{\textcolor[rgb]{0.40,0.40,0.40}{##1}}}
\@namedef{PY@tok@mo}{\def\PY@tc##1{\textcolor[rgb]{0.40,0.40,0.40}{##1}}}
\@namedef{PY@tok@ch}{\let\PY@it=\textit\def\PY@tc##1{\textcolor[rgb]{0.25,0.50,0.50}{##1}}}
\@namedef{PY@tok@cm}{\let\PY@it=\textit\def\PY@tc##1{\textcolor[rgb]{0.25,0.50,0.50}{##1}}}
\@namedef{PY@tok@cpf}{\let\PY@it=\textit\def\PY@tc##1{\textcolor[rgb]{0.25,0.50,0.50}{##1}}}
\@namedef{PY@tok@c1}{\let\PY@it=\textit\def\PY@tc##1{\textcolor[rgb]{0.25,0.50,0.50}{##1}}}
\@namedef{PY@tok@cs}{\let\PY@it=\textit\def\PY@tc##1{\textcolor[rgb]{0.25,0.50,0.50}{##1}}}

\def\PYZbs{\char`\\}
\def\PYZus{\char`\_}
\def\PYZob{\char`\{}
\def\PYZcb{\char`\}}
\def\PYZca{\char`\^}
\def\PYZam{\char`\&}
\def\PYZlt{\char`\<}
\def\PYZgt{\char`\>}
\def\PYZsh{\char`\#}
\def\PYZpc{\char`\%}
\def\PYZdl{\char`\$}
\def\PYZhy{\char`\-}
\def\PYZsq{\char`\'}
\def\PYZdq{\char`\"}
\def\PYZti{\char`\~}
% for compatibility with earlier versions
\def\PYZat{@}
\def\PYZlb{[}
\def\PYZrb{]}
\makeatother


    % For linebreaks inside Verbatim environment from package fancyvrb. 
    \makeatletter
        \newbox\Wrappedcontinuationbox 
        \newbox\Wrappedvisiblespacebox 
        \newcommand*\Wrappedvisiblespace {\textcolor{red}{\textvisiblespace}} 
        \newcommand*\Wrappedcontinuationsymbol {\textcolor{red}{\llap{\tiny$\m@th\hookrightarrow$}}} 
        \newcommand*\Wrappedcontinuationindent {3ex } 
        \newcommand*\Wrappedafterbreak {\kern\Wrappedcontinuationindent\copy\Wrappedcontinuationbox} 
        % Take advantage of the already applied Pygments mark-up to insert 
        % potential linebreaks for TeX processing. 
        %        {, <, #, %, $, ' and ": go to next line. 
        %        _, }, ^, &, >, - and ~: stay at end of broken line. 
        % Use of \textquotesingle for straight quote. 
        \newcommand*\Wrappedbreaksatspecials {% 
            \def\PYGZus{\discretionary{\char`\_}{\Wrappedafterbreak}{\char`\_}}% 
            \def\PYGZob{\discretionary{}{\Wrappedafterbreak\char`\{}{\char`\{}}% 
            \def\PYGZcb{\discretionary{\char`\}}{\Wrappedafterbreak}{\char`\}}}% 
            \def\PYGZca{\discretionary{\char`\^}{\Wrappedafterbreak}{\char`\^}}% 
            \def\PYGZam{\discretionary{\char`\&}{\Wrappedafterbreak}{\char`\&}}% 
            \def\PYGZlt{\discretionary{}{\Wrappedafterbreak\char`\<}{\char`\<}}% 
            \def\PYGZgt{\discretionary{\char`\>}{\Wrappedafterbreak}{\char`\>}}% 
            \def\PYGZsh{\discretionary{}{\Wrappedafterbreak\char`\#}{\char`\#}}% 
            \def\PYGZpc{\discretionary{}{\Wrappedafterbreak\char`\%}{\char`\%}}% 
            \def\PYGZdl{\discretionary{}{\Wrappedafterbreak\char`\$}{\char`\$}}% 
            \def\PYGZhy{\discretionary{\char`\-}{\Wrappedafterbreak}{\char`\-}}% 
            \def\PYGZsq{\discretionary{}{\Wrappedafterbreak\textquotesingle}{\textquotesingle}}% 
            \def\PYGZdq{\discretionary{}{\Wrappedafterbreak\char`\"}{\char`\"}}% 
            \def\PYGZti{\discretionary{\char`\~}{\Wrappedafterbreak}{\char`\~}}% 
        } 
        % Some characters . , ; ? ! / are not pygmentized. 
        % This macro makes them "active" and they will insert potential linebreaks 
        \newcommand*\Wrappedbreaksatpunct {% 
            \lccode`\~`\.\lowercase{\def~}{\discretionary{\hbox{\char`\.}}{\Wrappedafterbreak}{\hbox{\char`\.}}}% 
            \lccode`\~`\,\lowercase{\def~}{\discretionary{\hbox{\char`\,}}{\Wrappedafterbreak}{\hbox{\char`\,}}}% 
            \lccode`\~`\;\lowercase{\def~}{\discretionary{\hbox{\char`\;}}{\Wrappedafterbreak}{\hbox{\char`\;}}}% 
            \lccode`\~`\:\lowercase{\def~}{\discretionary{\hbox{\char`\:}}{\Wrappedafterbreak}{\hbox{\char`\:}}}% 
            \lccode`\~`\?\lowercase{\def~}{\discretionary{\hbox{\char`\?}}{\Wrappedafterbreak}{\hbox{\char`\?}}}% 
            \lccode`\~`\!\lowercase{\def~}{\discretionary{\hbox{\char`\!}}{\Wrappedafterbreak}{\hbox{\char`\!}}}% 
            \lccode`\~`\/\lowercase{\def~}{\discretionary{\hbox{\char`\/}}{\Wrappedafterbreak}{\hbox{\char`\/}}}% 
            \catcode`\.\active
            \catcode`\,\active 
            \catcode`\;\active
            \catcode`\:\active
            \catcode`\?\active
            \catcode`\!\active
            \catcode`\/\active 
            \lccode`\~`\~ 	
        }
    \makeatother

    \let\OriginalVerbatim=\Verbatim
    \makeatletter
    \renewcommand{\Verbatim}[1][1]{%
        %\parskip\z@skip
        \sbox\Wrappedcontinuationbox {\Wrappedcontinuationsymbol}%
        \sbox\Wrappedvisiblespacebox {\FV@SetupFont\Wrappedvisiblespace}%
        \def\FancyVerbFormatLine ##1{\hsize\linewidth
            \vtop{\raggedright\hyphenpenalty\z@\exhyphenpenalty\z@
                \doublehyphendemerits\z@\finalhyphendemerits\z@
                \strut ##1\strut}%
        }%
        % If the linebreak is at a space, the latter will be displayed as visible
        % space at end of first line, and a continuation symbol starts next line.
        % Stretch/shrink are however usually zero for typewriter font.
        \def\FV@Space {%
            \nobreak\hskip\z@ plus\fontdimen3\font minus\fontdimen4\font
            \discretionary{\copy\Wrappedvisiblespacebox}{\Wrappedafterbreak}
            {\kern\fontdimen2\font}%
        }%
        
        % Allow breaks at special characters using \PYG... macros.
        \Wrappedbreaksatspecials
        % Breaks at punctuation characters . , ; ? ! and / need catcode=\active 	
        \OriginalVerbatim[#1,codes*=\Wrappedbreaksatpunct]%
    }
    \makeatother

    % Exact colors from NB
    \definecolor{incolor}{HTML}{303F9F}
    \definecolor{outcolor}{HTML}{D84315}
    \definecolor{cellborder}{HTML}{CFCFCF}
    \definecolor{cellbackground}{HTML}{F7F7F7}
    
    % prompt
    \makeatletter
    \newcommand{\boxspacing}{\kern\kvtcb@left@rule\kern\kvtcb@boxsep}
    \makeatother
    \newcommand{\prompt}[4]{
        {\ttfamily\llap{{\color{#2}[#3]:\hspace{3pt}#4}}\vspace{-\baselineskip}}
    }
    

    
    % Prevent overflowing lines due to hard-to-break entities
    \sloppy 
    % Setup hyperref package
    \hypersetup{
      breaklinks=true,  % so long urls are correctly broken across lines
      colorlinks=true,
      urlcolor=urlcolor,
      linkcolor=linkcolor,
      citecolor=citecolor,
      }
    % Slightly bigger margins than the latex defaults
    
    \geometry{verbose,tmargin=1in,bmargin=1in,lmargin=1in,rmargin=1in}
    
    

\begin{document}
    
    \maketitle
    
    

    
    \hypertarget{standard-input-and-output}{%
\section{Standard Input and Output}\label{standard-input-and-output}}

\hypertarget{topics}{%
\subsection{Topics}\label{topics}}

\begin{itemize}
\tightlist
\item
  common way to input and output data
\item
  printing variables and values onto monitor or console
\item
  reading data from keyboard
\item
  composing programs
\end{itemize}

    \hypertarget{input-and-output-io}{%
\subsection{Input and output (IO)}\label{input-and-output-io}}

\begin{itemize}
\tightlist
\item
  IO operations are fundamental to computer programs
\item
  C++ IO occurs in streams (sequence of bytes)
\item
  programs must be able to read data from varieties of input devices
  (\textbf{input operation})

  \begin{itemize}
  \tightlist
  \item
    streams of bytes flow from keyboard, disk drive, network devices,
    etc. to main memory, RAM (Random Access Memory)
  \end{itemize}
\item
  programs must be able to write data to varieties of output devices
  (\textbf{output operation})

  \begin{itemize}
  \tightlist
  \item
    stream of bytes flow from RAM to monitor, disk drive, network
    devices, etc.
  \end{itemize}
\item
  this chapter covers how C++ handle standard input and output
\item
  reading from and writing to disk drive or files is covered in File IO
  chapter
\end{itemize}

    \hypertarget{standard-output-stream}{%
\subsection{Standard output stream}\label{standard-output-stream}}

\begin{itemize}
\item
  a prgram may need to display data or results of computation to users
\item
  a common way to display results is by printing them to common output
  called monitor

  \begin{itemize}
  \tightlist
  \item
    also called console
  \end{itemize}
\item
  we've printed \texttt{hello\ world} and some other strings to console
  in Chapter 1
\item
  similarly, we can print any literal values or data stored in variables
  to standard output
\item
  use \texttt{cout} statement defined in
  \texttt{\textless{}iostream\textgreater{}} library and
  \texttt{std\ namespace}
\item
  output statement syntax:

\begin{Shaded}
\begin{Highlighting}[]
\NormalTok{cout }\OperatorTok{\textless{}\textless{}}\NormalTok{ varName1 }\OperatorTok{\textless{}\textless{}}\NormalTok{ varName2 }\OperatorTok{\textless{}\textless{}} \StringTok{"literal values"} \OperatorTok{\textless{}\textless{}} \CharTok{\textquotesingle{} \textquotesingle{}} \OperatorTok{\textless{}\textless{}} \DecValTok{100} \OperatorTok{\textless{}\textless{}} \CharTok{\textquotesingle{}}\SpecialCharTok{\textbackslash{}n}\CharTok{\textquotesingle{}}\OperatorTok{;}
\end{Highlighting}
\end{Shaded}
\item
  \texttt{\textless{}\textless{}} \textbf{stream insertion operator}
  inserts values to output stream
\item
  multiple values are separated by \texttt{\textless{}\textless{}}
  operator
\item
  \texttt{endl} operator or \texttt{\textbackslash{}n} escape character
  end line to continue writing in next line
\item
  the following code demonstrates standard output stream
\end{itemize}

    \begin{tcolorbox}[breakable, size=fbox, boxrule=1pt, pad at break*=1mm,colback=cellbackground, colframe=cellborder]
\prompt{In}{incolor}{1}{\boxspacing}
\begin{Verbatim}[commandchars=\\\{\}]
\PY{c+c1}{// include required library}
\PY{c+cp}{\PYZsh{}}\PY{c+cp}{include} \PY{c+cpf}{\PYZlt{}iostream\PYZgt{}}\PY{c+c1}{ // cout}

\PY{c+c1}{// use required namespace}
\PY{k}{using} \PY{k}{namespace} \PY{n+nn}{std}\PY{p}{;} \PY{c+c1}{//std namespace defines cout, endl, etc.}
\end{Verbatim}
\end{tcolorbox}

    \begin{tcolorbox}[breakable, size=fbox, boxrule=1pt, pad at break*=1mm,colback=cellbackground, colframe=cellborder]
\prompt{In}{incolor}{3}{\boxspacing}
\begin{Verbatim}[commandchars=\\\{\}]
\PY{n}{cout} \PY{o}{\PYZlt{}}\PY{o}{\PYZlt{}} \PY{l+s}{\PYZdq{}}\PY{l+s}{Hello World!}\PY{l+s}{\PYZdq{}} \PY{o}{\PYZlt{}}\PY{o}{\PYZlt{}} \PY{n}{endl}\PY{p}{;}
\PY{n}{cout} \PY{o}{\PYZlt{}}\PY{o}{\PYZlt{}} \PY{l+m+mi}{100} \PY{o}{\PYZlt{}}\PY{o}{\PYZlt{}} \PY{l+m+mf}{2.5f} \PY{o}{\PYZlt{}}\PY{o}{\PYZlt{}} \PY{l+s+sc}{\PYZsq{}}\PY{l+s+sc}{ }\PY{l+s+sc}{\PYZsq{}} \PY{o}{\PYZlt{}}\PY{o}{\PYZlt{}} \PY{l+m+mf}{3.99} \PY{o}{\PYZlt{}}\PY{o}{\PYZlt{}} \PY{l+s+sc}{\PYZsq{}}\PY{l+s+sc}{A}\PY{l+s+sc}{\PYZsq{}} \PY{o}{\PYZlt{}}\PY{o}{\PYZlt{}} \PY{l+s}{\PYZdq{}}\PY{l+s}{some text as string}\PY{l+s}{\PYZdq{}}\PY{p}{;}
\PY{n}{cout} \PY{o}{\PYZlt{}}\PY{o}{\PYZlt{}} \PY{l+s}{\PYZdq{}}\PY{l+s}{continue printing stuff in next line...?}\PY{l+s}{\PYZdq{}} \PY{o}{\PYZlt{}}\PY{o}{\PYZlt{}} \PY{n}{endl}\PY{p}{;}
\end{Verbatim}
\end{tcolorbox}

    \begin{Verbatim}[commandchars=\\\{\}]
Hello World!
1002.5 3.99Asome text as stringcontinue printing stuff in next line{\ldots}?
    \end{Verbatim}

    \begin{tcolorbox}[breakable, size=fbox, boxrule=1pt, pad at break*=1mm,colback=cellbackground, colframe=cellborder]
\prompt{In}{incolor}{4}{\boxspacing}
\begin{Verbatim}[commandchars=\\\{\}]
\PY{c+c1}{// declaring and printing variables}
\PY{c+cp}{\PYZsh{}}\PY{c+cp}{include} \PY{c+cpf}{\PYZlt{}string\PYZgt{}}

\PY{n}{string} \PY{n}{name} \PY{o}{=} \PY{l+s}{\PYZdq{}}\PY{l+s}{John Doe}\PY{l+s}{\PYZdq{}}\PY{p}{;}
\PY{k+kt}{char} \PY{n}{MI} \PY{o}{=} \PY{l+s+sc}{\PYZsq{}}\PY{l+s+sc}{A}\PY{l+s+sc}{\PYZsq{}}\PY{p}{;}
\PY{k+kt}{int} \PY{n}{age} \PY{o}{=} \PY{l+m+mi}{25}\PY{p}{;}
\end{Verbatim}
\end{tcolorbox}

    \begin{tcolorbox}[breakable, size=fbox, boxrule=1pt, pad at break*=1mm,colback=cellbackground, colframe=cellborder]
\prompt{In}{incolor}{5}{\boxspacing}
\begin{Verbatim}[commandchars=\\\{\}]
\PY{c+c1}{// outputting variables}
\PY{n}{cout} \PY{o}{\PYZlt{}}\PY{o}{\PYZlt{}} \PY{l+s}{\PYZdq{}}\PY{l+s}{name = }\PY{l+s}{\PYZdq{}} \PY{o}{\PYZlt{}}\PY{o}{\PYZlt{}} \PY{n}{name} \PY{o}{\PYZlt{}}\PY{o}{\PYZlt{}} \PY{n}{endl}\PY{p}{;}
\PY{n}{cout} \PY{o}{\PYZlt{}}\PY{o}{\PYZlt{}} \PY{l+s}{\PYZdq{}}\PY{l+s}{MI = }\PY{l+s}{\PYZdq{}} \PY{o}{\PYZlt{}}\PY{o}{\PYZlt{}} \PY{n}{MI} \PY{o}{\PYZlt{}}\PY{o}{\PYZlt{}} \PY{l+s}{\PYZdq{}}\PY{l+s}{ and age = }\PY{l+s}{\PYZdq{}} \PY{o}{\PYZlt{}}\PY{o}{\PYZlt{}} \PY{n}{age} \PY{o}{\PYZlt{}}\PY{o}{\PYZlt{}} \PY{n}{endl}\PY{p}{;}
\end{Verbatim}
\end{tcolorbox}

    \begin{Verbatim}[commandchars=\\\{\}]
name = John Doe
MI = A and age = 25
    \end{Verbatim}

    \begin{tcolorbox}[breakable, size=fbox, boxrule=1pt, pad at break*=1mm,colback=cellbackground, colframe=cellborder]
\prompt{In}{incolor}{6}{\boxspacing}
\begin{Verbatim}[commandchars=\\\{\}]
\PY{k+kt}{bool} \PY{n}{done} \PY{o}{=} \PY{n+nb}{false}\PY{p}{;}
\PY{k+kt}{float} \PY{n}{temperature} \PY{o}{=} \PY{l+m+mi}{73}\PY{p}{;}
\PY{k+kt}{float} \PY{n}{richest\PYZus{}persons\PYZus{}networth} \PY{o}{=} \PY{l+m+mi}{120000000000}\PY{p}{;} \PY{c+c1}{// 120 billion}
\PY{k+kt}{float} \PY{n}{interestRate} \PY{o}{=} \PY{l+m+mf}{4.5}\PY{p}{;}
\PY{k+kt}{float} \PY{n}{length} \PY{o}{=} \PY{l+m+mf}{10.5}\PY{p}{;}
\PY{k+kt}{float} \PY{n}{width} \PY{o}{=} \PY{l+m+mf}{99.99f}\PY{p}{;} \PY{c+c1}{// can end with f for representing floating point number}
\PY{k+kt}{double} \PY{n}{space\PYZus{}shuttle\PYZus{}velocity} \PY{o}{=} \PY{l+m+mf}{950.1234567891234567} \PY{c+c1}{// 16 decimal points}
\end{Verbatim}
\end{tcolorbox}

    \begin{tcolorbox}[breakable, size=fbox, boxrule=1pt, pad at break*=1mm,colback=cellbackground, colframe=cellborder]
\prompt{In}{incolor}{7}{\boxspacing}
\begin{Verbatim}[commandchars=\\\{\}]
\PY{c+c1}{// cout can continue in multilines}
\PY{n}{cout} \PY{o}{\PYZlt{}}\PY{o}{\PYZlt{}} \PY{l+s}{\PYZdq{}}\PY{l+s}{temperature: }\PY{l+s}{\PYZdq{}} \PY{o}{\PYZlt{}}\PY{o}{\PYZlt{}} \PY{n}{temperature} \PY{o}{\PYZlt{}}\PY{o}{\PYZlt{}} \PY{l+s}{\PYZdq{}}\PY{l+s}{ age: }\PY{l+s}{\PYZdq{}} \PY{o}{\PYZlt{}}\PY{o}{\PYZlt{}} \PY{n}{age} 
    \PY{o}{\PYZlt{}}\PY{o}{\PYZlt{}} \PY{l+s}{\PYZdq{}}\PY{l+s}{ richest person\PYZsq{}s worth: }\PY{l+s}{\PYZdq{}}
    \PY{o}{\PYZlt{}}\PY{o}{\PYZlt{}} \PY{n}{richest\PYZus{}persons\PYZus{}networth} \PY{o}{\PYZlt{}}\PY{o}{\PYZlt{}} \PY{n}{endl}\PY{p}{;}
\PY{n}{cout} \PY{o}{\PYZlt{}}\PY{o}{\PYZlt{}} \PY{l+s}{\PYZdq{}}\PY{l+s}{interest rate: }\PY{l+s}{\PYZdq{}} \PY{o}{\PYZlt{}}\PY{o}{\PYZlt{}} \PY{n}{interestRate} \PY{o}{\PYZlt{}}\PY{o}{\PYZlt{}} \PY{n}{endl}\PY{p}{;}
\PY{n}{cout} \PY{o}{\PYZlt{}}\PY{o}{\PYZlt{}} \PY{l+s}{\PYZdq{}}\PY{l+s}{length: }\PY{l+s}{\PYZdq{}} \PY{o}{\PYZlt{}}\PY{o}{\PYZlt{}} \PY{n}{length} \PY{o}{\PYZlt{}}\PY{o}{\PYZlt{}} \PY{l+s}{\PYZdq{}}\PY{l+s}{ and width = }\PY{l+s}{\PYZdq{}} \PY{o}{\PYZlt{}}\PY{o}{\PYZlt{}} \PY{n}{width} \PY{o}{\PYZlt{}}\PY{o}{\PYZlt{}} \PY{n}{endl}\PY{p}{;}
\PY{n}{cout} \PY{o}{\PYZlt{}}\PY{o}{\PYZlt{}} \PY{l+s}{\PYZdq{}}\PY{l+s}{space\PYZus{}shuttle\PYZus{}velocity: }\PY{l+s}{\PYZdq{}} \PY{o}{\PYZlt{}}\PY{o}{\PYZlt{}} \PY{l+m+mf}{950.1234567891234567} \PY{o}{\PYZlt{}}\PY{o}{\PYZlt{}} \PY{n}{endl}\PY{p}{;}
\end{Verbatim}
\end{tcolorbox}

    \begin{Verbatim}[commandchars=\\\{\}]
temperature: 73 age: 25 richest person's worth: 1.2e+11
interest rate: 4.5
length: 10.5 and width = 99.99
space\_shuttle\_velocity: 950.123
    \end{Verbatim}

    \begin{tcolorbox}[breakable, size=fbox, boxrule=1pt, pad at break*=1mm,colback=cellbackground, colframe=cellborder]
\prompt{In}{incolor}{8}{\boxspacing}
\begin{Verbatim}[commandchars=\\\{\}]
\PY{c+c1}{// outputting string variables}
\PY{n}{cout} \PY{o}{\PYZlt{}}\PY{o}{\PYZlt{}} \PY{l+s}{\PYZdq{}}\PY{l+s}{Hello there, }\PY{l+s}{\PYZdq{}} \PY{o}{\PYZlt{}}\PY{o}{\PYZlt{}} \PY{n}{name} \PY{o}{\PYZlt{}}\PY{o}{\PYZlt{}} \PY{l+s+sc}{\PYZsq{}}\PY{l+s+sc}{!}\PY{l+s+sc}{\PYZsq{}} \PY{o}{\PYZlt{}}\PY{o}{\PYZlt{}} \PY{n}{endl}\PY{p}{;}
\end{Verbatim}
\end{tcolorbox}

    \begin{Verbatim}[commandchars=\\\{\}]
Hello there, John Doe!
    \end{Verbatim}

    \begin{tcolorbox}[breakable, size=fbox, boxrule=1pt, pad at break*=1mm,colback=cellbackground, colframe=cellborder]
\prompt{In}{incolor}{9}{\boxspacing}
\begin{Verbatim}[commandchars=\\\{\}]
\PY{c+c1}{// more string variables}
\PY{n}{string} \PY{n}{address1} \PY{o}{=} \PY{l+s}{\PYZdq{}}\PY{l+s}{1100 North Ave}\PY{l+s}{\PYZdq{}}\PY{p}{;}
\PY{n}{string} \PY{n}{state\PYZus{}code} \PY{o}{=} \PY{l+s}{\PYZdq{}}\PY{l+s}{CO}\PY{l+s}{\PYZdq{}}\PY{p}{;}
\PY{n}{string} \PY{n}{country} \PY{o}{=} \PY{l+s}{\PYZdq{}}\PY{l+s}{USA}\PY{l+s}{\PYZdq{}}\PY{p}{;}
\end{Verbatim}
\end{tcolorbox}

    \begin{tcolorbox}[breakable, size=fbox, boxrule=1pt, pad at break*=1mm,colback=cellbackground, colframe=cellborder]
\prompt{In}{incolor}{10}{\boxspacing}
\begin{Verbatim}[commandchars=\\\{\}]
\PY{n}{cout} \PY{o}{\PYZlt{}}\PY{o}{\PYZlt{}} \PY{l+s}{\PYZdq{}}\PY{l+s}{CMU\PYZsq{}s address:}\PY{l+s+se}{\PYZbs{}n}\PY{l+s}{\PYZdq{}}
     \PY{o}{\PYZlt{}}\PY{o}{\PYZlt{}} \PY{n}{address1} \PY{o}{\PYZlt{}}\PY{o}{\PYZlt{}} \PY{n}{endl}
     \PY{o}{\PYZlt{}}\PY{o}{\PYZlt{}} \PY{l+s}{\PYZdq{}}\PY{l+s}{Grand Junction, }\PY{l+s}{\PYZdq{}} \PY{o}{\PYZlt{}}\PY{o}{\PYZlt{}} \PY{n}{state\PYZus{}code} \PY{o}{\PYZlt{}}\PY{o}{\PYZlt{}} \PY{l+s+sc}{\PYZsq{}}\PY{l+s+sc}{ }\PY{l+s+sc}{\PYZsq{}} \PY{o}{\PYZlt{}}\PY{o}{\PYZlt{}} \PY{l+m+mi}{81501} \PY{o}{\PYZlt{}}\PY{o}{\PYZlt{}} \PY{n}{endl}
     \PY{o}{\PYZlt{}}\PY{o}{\PYZlt{}} \PY{n}{country} \PY{o}{\PYZlt{}}\PY{o}{\PYZlt{}} \PY{n}{endl}\PY{p}{;}
\end{Verbatim}
\end{tcolorbox}

    \begin{Verbatim}[commandchars=\\\{\}]
CMU's address:
1100 North Ave
Grand Junction, CO 81501
USA
    \end{Verbatim}

    \hypertarget{escape-sequences}{%
\subsubsection{Escape sequences}\label{escape-sequences}}

\begin{itemize}
\tightlist
\item
  some letters or sequence of letters have special meaning to C++

  \begin{itemize}
  \tightlist
  \item
    e.g., pair of single quote is used to represent a character data,
    e.g.~`A' or ' ' (space)
  \item
    and pair of double quotes is used to represent a string type, e.g.,
    ``Hello World!''
  \end{itemize}
\item
  how can we store single or double quotes as part of data?

  \begin{itemize}
  \tightlist
  \item
    e.g., we need to print: \textbf{``Oh no!'', Alice exclaimed, ``Bob's
    bike is broken!''}
  \item
    we use backslash \texttt{\textbackslash{}} (escape character) to
    escape the special meaning of single and double quotes or other
    characters
  \end{itemize}
\item
  characters represented using escape character are called escape
  sequences

  \begin{itemize}
  \tightlist
  \item
    \textbackslash n - new line
  \item
    \textbackslash\textbackslash{} - back slash
  \item
    \textbackslash t - tab
  \item
    \textbackslash r - carriage return
  \item
    \textbackslash' - single quote
  \item
    \textbackslash'' - double quote
  \end{itemize}
\end{itemize}

    \begin{tcolorbox}[breakable, size=fbox, boxrule=1pt, pad at break*=1mm,colback=cellbackground, colframe=cellborder]
\prompt{In}{incolor}{11}{\boxspacing}
\begin{Verbatim}[commandchars=\\\{\}]
\PY{n}{cout} \PY{o}{\PYZlt{}}\PY{o}{\PYZlt{}} \PY{l+s}{\PYZdq{}}\PY{l+s}{What\PYZsq{}s up}\PY{l+s+se}{\PYZbs{}n}\PY{l+s}{ Shaq}\PY{l+s+se}{\PYZbs{}t}\PY{l+s}{O\PYZsq{}Neal?}\PY{l+s}{\PYZdq{}}\PY{p}{;}
\end{Verbatim}
\end{tcolorbox}

    \begin{Verbatim}[commandchars=\\\{\}]
What's up
 Shaq   O'Neal?
    \end{Verbatim}

    \begin{tcolorbox}[breakable, size=fbox, boxrule=1pt, pad at break*=1mm,colback=cellbackground, colframe=cellborder]
\prompt{In}{incolor}{12}{\boxspacing}
\begin{Verbatim}[commandchars=\\\{\}]
\PY{k+kt}{char} \PY{n}{quote} \PY{o}{=} \PY{l+s+sc}{\PYZsq{}}\PY{l+s+sc}{\PYZbs{}\PYZsq{}}\PY{l+s+sc}{\PYZsq{}}\PY{p}{;}
\end{Verbatim}
\end{tcolorbox}

    \begin{tcolorbox}[breakable, size=fbox, boxrule=1pt, pad at break*=1mm,colback=cellbackground, colframe=cellborder]
\prompt{In}{incolor}{13}{\boxspacing}
\begin{Verbatim}[commandchars=\\\{\}]
\PY{n}{quote}
\end{Verbatim}
\end{tcolorbox}

            \begin{tcolorbox}[breakable, size=fbox, boxrule=.5pt, pad at break*=1mm, opacityfill=0]
\prompt{Out}{outcolor}{13}{\boxspacing}
\begin{Verbatim}[commandchars=\\\{\}]
'''
\end{Verbatim}
\end{tcolorbox}
        
    \begin{tcolorbox}[breakable, size=fbox, boxrule=1pt, pad at break*=1mm,colback=cellbackground, colframe=cellborder]
\prompt{In}{incolor}{14}{\boxspacing}
\begin{Verbatim}[commandchars=\\\{\}]
\PY{n}{cout} \PY{o}{\PYZlt{}}\PY{o}{\PYZlt{}} \PY{l+s}{\PYZdq{}}\PY{l+s+se}{\PYZbs{}\PYZdq{}}\PY{l+s}{Oh no!}\PY{l+s+se}{\PYZbs{}\PYZdq{}}\PY{l+s}{, Alice exclaimed, }\PY{l+s+se}{\PYZbs{}\PYZdq{}}\PY{l+s}{Bob\PYZsq{}s bike is broken!}\PY{l+s+se}{\PYZbs{}\PYZdq{}}\PY{l+s}{\PYZdq{}}\PY{p}{;}
\end{Verbatim}
\end{tcolorbox}

    \begin{Verbatim}[commandchars=\\\{\}]
"Oh no!", Alice exclaimed, "Bob's bike is broken!"
    \end{Verbatim}

    \begin{tcolorbox}[breakable, size=fbox, boxrule=1pt, pad at break*=1mm,colback=cellbackground, colframe=cellborder]
\prompt{In}{incolor}{2}{\boxspacing}
\begin{Verbatim}[commandchars=\\\{\}]
\PY{n}{cout} \PY{o}{\PYZlt{}}\PY{o}{\PYZlt{}} \PY{l+s}{\PYZdq{}}\PY{l+s}{how many back slashes will be printed? }\PY{l+s+se}{\PYZbs{}\PYZbs{}}\PY{l+s+se}{\PYZbs{}\PYZbs{}}\PY{l+s}{\PYZdq{}}\PY{p}{;}
\end{Verbatim}
\end{tcolorbox}

    \begin{Verbatim}[commandchars=\\\{\}]
how many back slashes will be printed? \textbackslash{}\textbackslash{}
    \end{Verbatim}

    \hypertarget{standard-input-stream}{%
\subsection{Standard input stream}\label{standard-input-stream}}

\begin{itemize}
\item
  often, data must be read from standard input stream or keyboard

  \begin{itemize}
  \tightlist
  \item
    e.g.~most interactive programs with Graphical User Interface (GUI)
    or Command Line Interface (CLI)
  \end{itemize}
\item
  must include \texttt{\textless{}iostream\textgreater{}} library for
  standard input
\item
  must use \textbf{std} namespace
\item
  use \textbf{cin \textgreater\textgreater{}} statement
\item
  syntax:

\begin{Shaded}
\begin{Highlighting}[]
\NormalTok{cin }\OperatorTok{\textgreater{}\textgreater{}}\NormalTok{ var1 }\OperatorTok{\textgreater{}\textgreater{}}\NormalTok{ var2 }\OperatorTok{\textgreater{}\textgreater{}} \OperatorTok{...;}
\end{Highlighting}
\end{Shaded}
\item
  \texttt{\textgreater{}\textgreater{}} stream extraction operator
  extracts data/value from input stream
\item
  must always use variables of appropriate types
\item
  while scanning input stream, \texttt{\textgreater{}\textgreater{}}
  ignores leading whitespaces and stops at a trailing whitespace
\item
  let's say we have a stream of data separated by whitespaces:
  \texttt{10\ 11\ 15.5\ A}

  \begin{itemize}
  \tightlist
  \item
    we can parse and extract it as following:
  \end{itemize}

\begin{Shaded}
\begin{Highlighting}[]
\NormalTok{cin }\OperatorTok{\textgreater{}\textgreater{}}\NormalTok{ num1 }\OperatorTok{\textgreater{}\textgreater{}}\NormalTok{ num2 }\OperatorTok{\textgreater{}\textgreater{}}\NormalTok{ num3 }\OperatorTok{\textgreater{}\textgreater{}}\NormalTok{ alpha}\OperatorTok{;}
\end{Highlighting}
\end{Shaded}

\begin{verbatim}
  - given num1 and num2 are of type int or long, num3 is float or double and alpha is char
\end{verbatim}
\end{itemize}

\hypertarget{inputting-numerical-data}{%
\subsubsection{Inputting numerical
data}\label{inputting-numerical-data}}

\begin{itemize}
\tightlist
\item
  we must store the extracted numerical input data into appropriate
  numerical variables
\item
  \texttt{\textgreater{}\textgreater{}\ int\ variables} : extracts whole
  numbers from input stream; stops at anything else
\item
  \texttt{\textgreater{}\textgreater{}\ float\ or\ double\ variables} :
  extracts numbers including decimal points; stops at anything else
\end{itemize}

    \begin{tcolorbox}[breakable, size=fbox, boxrule=1pt, pad at break*=1mm,colback=cellbackground, colframe=cellborder]
\prompt{In}{incolor}{17}{\boxspacing}
\begin{Verbatim}[commandchars=\\\{\}]
\PY{c+c1}{// include required libraries}
\PY{c+cp}{\PYZsh{}}\PY{c+cp}{include} \PY{c+cpf}{\PYZlt{}iostream\PYZgt{}}\PY{c+c1}{ //cin, cout}

\PY{k}{using} \PY{k}{namespace} \PY{n+nn}{std}\PY{p}{;}
\end{Verbatim}
\end{tcolorbox}

    \begin{tcolorbox}[breakable, size=fbox, boxrule=1pt, pad at break*=1mm,colback=cellbackground, colframe=cellborder]
\prompt{In}{incolor}{15}{\boxspacing}
\begin{Verbatim}[commandchars=\\\{\}]
\PY{k+kt}{int} \PY{n}{num1}\PY{p}{;}
\PY{c+c1}{// prompt user to enter a whole number}
\PY{n}{cout} \PY{o}{\PYZlt{}}\PY{o}{\PYZlt{}} \PY{l+s}{\PYZdq{}}\PY{l+s}{enter a whole number: }\PY{l+s}{\PYZdq{}}\PY{p}{;}
\PY{n}{cin} \PY{o}{\PYZgt{}}\PY{o}{\PYZgt{}} \PY{n}{num1}\PY{p}{;}
\PY{n}{cout} \PY{o}{\PYZlt{}}\PY{o}{\PYZlt{}} \PY{l+s}{\PYZdq{}}\PY{l+s}{You entered: }\PY{l+s}{\PYZdq{}} \PY{o}{\PYZlt{}}\PY{o}{\PYZlt{}} \PY{n}{num1} \PY{o}{\PYZlt{}}\PY{o}{\PYZlt{}} \PY{n}{endl}\PY{p}{;}
\end{Verbatim}
\end{tcolorbox}

    \begin{Verbatim}[commandchars=\\\{\}]
enter a whole number: 10
You entered: 10
    \end{Verbatim}

    \begin{tcolorbox}[breakable, size=fbox, boxrule=1pt, pad at break*=1mm,colback=cellbackground, colframe=cellborder]
\prompt{In}{incolor}{10}{\boxspacing}
\begin{Verbatim}[commandchars=\\\{\}]
\PY{c+c1}{// can extract multiple integers}
\PY{k+kt}{int} \PY{n}{num2}\PY{p}{;}
\PY{n}{cout} \PY{o}{\PYZlt{}}\PY{o}{\PYZlt{}} \PY{l+s}{\PYZdq{}}\PY{l+s}{enter two whole numbers separated by space: }\PY{l+s}{\PYZdq{}}\PY{p}{;}
\PY{n}{cin} \PY{o}{\PYZgt{}}\PY{o}{\PYZgt{}} \PY{n}{num1} \PY{o}{\PYZgt{}}\PY{o}{\PYZgt{}} \PY{n}{num2}\PY{p}{;}
\PY{n}{cout} \PY{o}{\PYZlt{}}\PY{o}{\PYZlt{}} \PY{n}{num1} \PY{o}{\PYZlt{}}\PY{o}{\PYZlt{}} \PY{l+s+sc}{\PYZsq{}}\PY{l+s+sc}{+}\PY{l+s+sc}{\PYZsq{}} \PY{o}{\PYZlt{}}\PY{o}{\PYZlt{}} \PY{n}{num2} \PY{o}{\PYZlt{}}\PY{o}{\PYZlt{}} \PY{l+s+sc}{\PYZsq{}}\PY{l+s+sc}{=}\PY{l+s+sc}{\PYZsq{}} \PY{o}{\PYZlt{}}\PY{o}{\PYZlt{}} \PY{n}{num1}\PY{o}{+}\PY{n}{num2} \PY{o}{\PYZlt{}}\PY{o}{\PYZlt{}} \PY{n}{endl}\PY{p}{;}
\end{Verbatim}
\end{tcolorbox}

    \begin{Verbatim}[commandchars=\\\{\}]
enter two whole numbers separated by space: 10 20
10+20=30
    \end{Verbatim}

    \begin{tcolorbox}[breakable, size=fbox, boxrule=1pt, pad at break*=1mm,colback=cellbackground, colframe=cellborder]
\prompt{In}{incolor}{11}{\boxspacing}
\begin{Verbatim}[commandchars=\\\{\}]
\PY{c+c1}{// extracting int and float}
\PY{k+kt}{float} \PY{n}{num3}\PY{p}{;}
\PY{n}{cout} \PY{o}{\PYZlt{}}\PY{o}{\PYZlt{}} \PY{l+s}{\PYZdq{}}\PY{l+s}{enter a whole number and a floating point number separated by space: }\PY{l+s}{\PYZdq{}}\PY{p}{;}
\PY{n}{cin} \PY{o}{\PYZgt{}}\PY{o}{\PYZgt{}} \PY{n}{num1} \PY{o}{\PYZgt{}}\PY{o}{\PYZgt{}} \PY{n}{num3}\PY{p}{;}
\PY{n}{cout} \PY{o}{\PYZlt{}}\PY{o}{\PYZlt{}} \PY{n}{num1} \PY{o}{\PYZlt{}}\PY{o}{\PYZlt{}} \PY{l+s}{\PYZdq{}}\PY{l+s}{ + }\PY{l+s}{\PYZdq{}} \PY{o}{\PYZlt{}}\PY{o}{\PYZlt{}} \PY{n}{num3} \PY{o}{\PYZlt{}}\PY{o}{\PYZlt{}} \PY{l+s}{\PYZdq{}}\PY{l+s}{ = }\PY{l+s}{\PYZdq{}} \PY{o}{\PYZlt{}}\PY{o}{\PYZlt{}} \PY{n}{num1}\PY{o}{+}\PY{n}{num3} \PY{o}{\PYZlt{}}\PY{o}{\PYZlt{}} \PY{n}{endl}\PY{p}{;}
\end{Verbatim}
\end{tcolorbox}

    \begin{Verbatim}[commandchars=\\\{\}]
enter a whole number and a floating point number separated by space: 5 9.9
5 + 9.9 = 14.9
    \end{Verbatim}

    \begin{tcolorbox}[breakable, size=fbox, boxrule=1pt, pad at break*=1mm,colback=cellbackground, colframe=cellborder]
\prompt{In}{incolor}{12}{\boxspacing}
\begin{Verbatim}[commandchars=\\\{\}]
\PY{c+c1}{// let\PYZsq{}s enter 10 11 15.5 A and store them into corresponding variables}
\PY{k+kt}{int} \PY{n}{n1}\PY{p}{,} \PY{n}{n2}\PY{p}{;}
\PY{k+kt}{float} \PY{n}{n3}\PY{p}{;}
\PY{k+kt}{char} \PY{n}{alpha}\PY{p}{;}
\end{Verbatim}
\end{tcolorbox}

    \begin{tcolorbox}[breakable, size=fbox, boxrule=1pt, pad at break*=1mm,colback=cellbackground, colframe=cellborder]
\prompt{In}{incolor}{13}{\boxspacing}
\begin{Verbatim}[commandchars=\\\{\}]
\PY{c+c1}{// let\PYZsq{}s not prompt; but simply enter 10 11 15.5 A}
\PY{n}{cin} \PY{o}{\PYZgt{}}\PY{o}{\PYZgt{}} \PY{n}{n1} \PY{o}{\PYZgt{}}\PY{o}{\PYZgt{}} \PY{n}{n2} \PY{o}{\PYZgt{}}\PY{o}{\PYZgt{}} \PY{n}{n3} \PY{o}{\PYZgt{}}\PY{o}{\PYZgt{}} \PY{n}{alpha}\PY{p}{;}
\end{Verbatim}
\end{tcolorbox}

    \begin{Verbatim}[commandchars=\\\{\}]
10 11 15.5 A
    \end{Verbatim}

    \begin{tcolorbox}[breakable, size=fbox, boxrule=1pt, pad at break*=1mm,colback=cellbackground, colframe=cellborder]
\prompt{In}{incolor}{14}{\boxspacing}
\begin{Verbatim}[commandchars=\\\{\}]
\PY{c+c1}{// let\PYZsq{}s echo the entered values}
\PY{n}{cout} \PY{o}{\PYZlt{}}\PY{o}{\PYZlt{}} \PY{n}{n1} \PY{o}{\PYZlt{}}\PY{o}{\PYZlt{}} \PY{l+s}{\PYZdq{}}\PY{l+s}{ }\PY{l+s}{\PYZdq{}} \PY{o}{\PYZlt{}}\PY{o}{\PYZlt{}} \PY{n}{n2} \PY{o}{\PYZlt{}}\PY{o}{\PYZlt{}} \PY{l+s}{\PYZdq{}}\PY{l+s}{ }\PY{l+s}{\PYZdq{}} \PY{o}{\PYZlt{}}\PY{o}{\PYZlt{}} \PY{n}{n3} \PY{o}{\PYZlt{}}\PY{o}{\PYZlt{}} \PY{l+s}{\PYZdq{}}\PY{l+s}{ }\PY{l+s}{\PYZdq{}} \PY{o}{\PYZlt{}}\PY{o}{\PYZlt{}} \PY{n}{alpha}\PY{p}{;}
\end{Verbatim}
\end{tcolorbox}

    \begin{Verbatim}[commandchars=\\\{\}]
10 11 15.5 A
    \end{Verbatim}

    \hypertarget{input-failure}{%
\subsubsection{Input failure}\label{input-failure}}

\begin{itemize}
\tightlist
\item
  if input data and variable type mismatched, \texttt{cin} will not be
  able to extract the data from the stream

  \begin{itemize}
  \tightlist
  \item
    \texttt{cin} will enter into a fail state
  \item
    won't be able to extract data anymore
  \end{itemize}
\item
  Note: Jupyter Notebook may crash or simply not work as expected when
  input fails

  \begin{itemize}
  \tightlist
  \item
    if the Jupyter crashes or stops working, your must restart the
    Kernel: \texttt{Kernel\ -\textgreater{}\ Restart}
  \end{itemize}
\end{itemize}

    \begin{tcolorbox}[breakable, size=fbox, boxrule=1pt, pad at break*=1mm,colback=cellbackground, colframe=cellborder]
\prompt{In}{incolor}{15}{\boxspacing}
\begin{Verbatim}[commandchars=\\\{\}]
\PY{c+c1}{// variable to store whole number}
\PY{k+kt}{int} \PY{n}{number}\PY{p}{;}
\end{Verbatim}
\end{tcolorbox}

    \begin{tcolorbox}[breakable, size=fbox, boxrule=1pt, pad at break*=1mm,colback=cellbackground, colframe=cellborder]
\prompt{In}{incolor}{16}{\boxspacing}
\begin{Verbatim}[commandchars=\\\{\}]
\PY{n}{cout} \PY{o}{\PYZlt{}}\PY{o}{\PYZlt{}} \PY{l+s}{\PYZdq{}}\PY{l+s}{Enter a number: }\PY{l+s}{\PYZdq{}}\PY{p}{;}
\PY{n}{cin} \PY{o}{\PYZgt{}}\PY{o}{\PYZgt{}} \PY{n}{number}\PY{p}{;}
\PY{n}{cout} \PY{o}{\PYZlt{}}\PY{o}{\PYZlt{}} \PY{l+s}{\PYZdq{}}\PY{l+s}{You entered }\PY{l+s}{\PYZdq{}} \PY{o}{\PYZlt{}}\PY{o}{\PYZlt{}} \PY{n}{number}\PY{p}{;}
\PY{c+c1}{// Play with it:}
\PY{c+c1}{// try entering an integer then whole number and characters then characters and number, etc.}
\end{Verbatim}
\end{tcolorbox}

    \begin{Verbatim}[commandchars=\\\{\}]
Enter a number: adf
You entered 0
    \end{Verbatim}

            \begin{tcolorbox}[breakable, size=fbox, boxrule=.5pt, pad at break*=1mm, opacityfill=0]
\prompt{Out}{outcolor}{16}{\boxspacing}
\begin{Verbatim}[commandchars=\\\{\}]
@0x107733ec0
\end{Verbatim}
\end{tcolorbox}
        
    \hypertarget{inputting-string-data}{%
\subsubsection{Inputting string data}\label{inputting-string-data}}

\begin{itemize}
\item
  we can read string data in two ways depending on if the string has a
  space (phrase) or not (word)
\item
  string without space or single word can be extracted using
  \texttt{\textgreater{}\textgreater{}} stream extraction operator
\item
  a single string data or line with spaces must be extracted using
  \texttt{getline(\ )} function
\item
  reading syntax:

\begin{Shaded}
\begin{Highlighting}[]
\NormalTok{getline}\OperatorTok{(}\NormalTok{cin}\OperatorTok{,}\NormalTok{ strVar}\OperatorTok{);} \CommentTok{// reading a line from std input and storing into strVar}
\end{Highlighting}
\end{Shaded}
\item
  \texttt{getline()} reads the entire line including whitespaces
  including the \texttt{\textbackslash{}n} newline

  \begin{itemize}
  \tightlist
  \item
    newline is extracted from the input stream and discarded
  \end{itemize}
\end{itemize}

    \begin{tcolorbox}[breakable, size=fbox, boxrule=1pt, pad at break*=1mm,colback=cellbackground, colframe=cellborder]
\prompt{In}{incolor}{17}{\boxspacing}
\begin{Verbatim}[commandchars=\\\{\}]
\PY{n}{string} \PY{n}{player\PYZus{}name}\PY{p}{;}
\end{Verbatim}
\end{tcolorbox}

    \begin{tcolorbox}[breakable, size=fbox, boxrule=1pt, pad at break*=1mm,colback=cellbackground, colframe=cellborder]
\prompt{In}{incolor}{18}{\boxspacing}
\begin{Verbatim}[commandchars=\\\{\}]
\PY{n}{cout} \PY{o}{\PYZlt{}}\PY{o}{\PYZlt{}} \PY{l+s}{\PYZdq{}}\PY{l+s}{Enter your first name: }\PY{l+s}{\PYZdq{}}\PY{p}{;}
\PY{n}{cin} \PY{o}{\PYZgt{}}\PY{o}{\PYZgt{}} \PY{n}{player\PYZus{}name}\PY{p}{;}
\PY{n}{cout} \PY{o}{\PYZlt{}}\PY{o}{\PYZlt{}} \PY{l+s}{\PYZdq{}}\PY{l+s}{Hello there, }\PY{l+s}{\PYZdq{}} \PY{o}{\PYZlt{}}\PY{o}{\PYZlt{}} \PY{n}{player\PYZus{}name} \PY{o}{\PYZlt{}}\PY{o}{\PYZlt{}} \PY{n}{endl}\PY{p}{;}
\PY{c+c1}{// run it wih just firstname and then with fullname; notice the value of player\PYZus{}name}
\end{Verbatim}
\end{tcolorbox}

    \begin{Verbatim}[commandchars=\\\{\}]
Enter your first name:
John Smith
Hello there, John
    \end{Verbatim}

            \begin{tcolorbox}[breakable, size=fbox, boxrule=.5pt, pad at break*=1mm, opacityfill=0]
\prompt{Out}{outcolor}{18}{\boxspacing}
\begin{Verbatim}[commandchars=\\\{\}]
@0x107733ec0
\end{Verbatim}
\end{tcolorbox}
        
    \begin{tcolorbox}[breakable, size=fbox, boxrule=1pt, pad at break*=1mm,colback=cellbackground, colframe=cellborder]
\prompt{In}{incolor}{19}{\boxspacing}
\begin{Verbatim}[commandchars=\\\{\}]
\PY{c+c1}{// string with spaces}
\PY{n}{cout} \PY{o}{\PYZlt{}}\PY{o}{\PYZlt{}} \PY{l+s}{\PYZdq{}}\PY{l+s}{Enter your full name: }\PY{l+s}{\PYZdq{}}\PY{p}{;}
\PY{n}{getline}\PY{p}{(}\PY{n}{cin}\PY{p}{,} \PY{n}{player\PYZus{}name}\PY{p}{)}\PY{p}{;}
\PY{n}{cout} \PY{o}{\PYZlt{}}\PY{o}{\PYZlt{}} \PY{l+s}{\PYZdq{}}\PY{l+s}{Hello there, }\PY{l+s}{\PYZdq{}} \PY{o}{\PYZlt{}}\PY{o}{\PYZlt{}} \PY{n}{player\PYZus{}name} \PY{o}{\PYZlt{}}\PY{o}{\PYZlt{}} \PY{n}{endl}\PY{p}{;}
\end{Verbatim}
\end{tcolorbox}

    \begin{Verbatim}[commandchars=\\\{\}]
Enter your full name: John Smith
Hello there, John Smith
    \end{Verbatim}

    \hypertarget{note}{%
\subsubsection{Note}\label{note}}

\begin{itemize}
\tightlist
\item
  getline( ) reads, discards and stops at newline character
  (\texttt{\textbackslash{}n})
\item
  \texttt{\textgreater{}\textgreater{}} stops before the trailing
  newline character leaving it in the input stream
\item
  must explictly read and discard newline character if getline is used
  after \texttt{\textgreater{}\textgreater{}}
\item
  use \textbf{ws} whitespace manipulator

  \begin{itemize}
  \tightlist
  \item
    ws operator extracts as many whitespace characters as possible from
    the current position in the input stream
  \item
    extraction stops as soon as a non-whitespace character is found
  \end{itemize}

\begin{Shaded}
\begin{Highlighting}[]
\NormalTok{cin }\OperatorTok{\textgreater{}\textgreater{}}\NormalTok{ number }\OperatorTok{\textgreater{}\textgreater{}}\NormalTok{ ws}\OperatorTok{;}
\end{Highlighting}
\end{Shaded}

  \begin{itemize}
  \tightlist
  \item
    reads and discards whitespace(s) including
    \texttt{\textbackslash{}n} after number value in input stream
  \end{itemize}
\end{itemize}

\hypertarget{demo-program}{%
\subsubsection{demo program}\label{demo-program}}

\begin{itemize}
\tightlist
\item
  program that demonstrates the above caveat is found here
  \url{demos/stdio/demo1/main.cpp}
\end{itemize}

    \hypertarget{composition}{%
\subsection{Composition}\label{composition}}

\begin{itemize}
\tightlist
\item
  similar to composing an essay or music

  \begin{itemize}
  \tightlist
  \item
    start with basic elements and combine them to build something bigger
    and meaningful work
  \end{itemize}
\item
  we use the same basic principle of \textbf{composition} in programming

  \begin{itemize}
  \tightlist
  \item
    take small building blocks

    \begin{itemize}
    \tightlist
    \item
      variables, values, expressions (operators), statements (input,
      output), etc.
    \end{itemize}
  \item
    compose something meaningful or solve a problem
  \end{itemize}
\end{itemize}

\hypertarget{example-1-find-area-and-perimeter-of-a-rectangle}{%
\subsubsection{example 1: find area and perimeter of a
rectangle}\label{example-1-find-area-and-perimeter-of-a-rectangle}}

\begin{itemize}
\tightlist
\item
  algorithm steps:

  \begin{enumerate}
  \def\labelenumi{\arabic{enumi}.}
  \tightlist
  \item
    get values for length and width of a rectangle
  \item
    calculate area and perimeter using the following equations

    \begin{itemize}
    \tightlist
    \item
      area = length x width
    \item
      perimeter = 2 x (length + width)
    \end{itemize}
  \item
    display the results
  \end{enumerate}
\end{itemize}

    \begin{tcolorbox}[breakable, size=fbox, boxrule=1pt, pad at break*=1mm,colback=cellbackground, colframe=cellborder]
\prompt{In}{incolor}{2}{\boxspacing}
\begin{Verbatim}[commandchars=\\\{\}]
\PY{c+c1}{// ex.1 program}
\PY{c+c1}{// variables to store length and width}
\PY{k+kt}{float} \PY{n}{rect\PYZus{}length}\PY{p}{,} \PY{n}{rect\PYZus{}width}\PY{p}{;}
\end{Verbatim}
\end{tcolorbox}

    \begin{tcolorbox}[breakable, size=fbox, boxrule=1pt, pad at break*=1mm,colback=cellbackground, colframe=cellborder]
\prompt{In}{incolor}{3}{\boxspacing}
\begin{Verbatim}[commandchars=\\\{\}]
\PY{c+c1}{// step 1 get length and width values }
\PY{c+c1}{// a. can be hardcoded literal values}
\PY{n}{rect\PYZus{}length} \PY{o}{=} \PY{l+m+mf}{10.5}\PY{p}{;} \PY{c+c1}{//hardcoded}
\PY{n}{rect\PYZus{}width} \PY{o}{=} \PY{l+m+mf}{5.5}\PY{p}{;}
\end{Verbatim}
\end{tcolorbox}

    \begin{tcolorbox}[breakable, size=fbox, boxrule=1pt, pad at break*=1mm,colback=cellbackground, colframe=cellborder]
\prompt{In}{incolor}{6}{\boxspacing}
\begin{Verbatim}[commandchars=\\\{\}]
\PY{c+c1}{// step 1.b or can be read from std input}
\PY{n}{cout} \PY{o}{\PYZlt{}}\PY{o}{\PYZlt{}} \PY{l+s}{\PYZdq{}}\PY{l+s}{Enter length and width of a rectangle separated by space: }\PY{l+s}{\PYZdq{}}\PY{p}{;}
\PY{n}{cin} \PY{o}{\PYZgt{}}\PY{o}{\PYZgt{}} \PY{n}{rect\PYZus{}length} \PY{o}{\PYZgt{}}\PY{o}{\PYZgt{}} \PY{n}{rect\PYZus{}width}\PY{p}{;}
\end{Verbatim}
\end{tcolorbox}

    \begin{Verbatim}[commandchars=\\\{\}]
Enter length and width of a rectangle separated by space: 11.2 6.6
    \end{Verbatim}

    \begin{tcolorbox}[breakable, size=fbox, boxrule=1pt, pad at break*=1mm,colback=cellbackground, colframe=cellborder]
\prompt{In}{incolor}{7}{\boxspacing}
\begin{Verbatim}[commandchars=\\\{\}]
\PY{n}{cout} \PY{o}{\PYZlt{}}\PY{o}{\PYZlt{}} \PY{l+s}{\PYZdq{}}\PY{l+s}{Rectangle\PYZsq{}s length = }\PY{l+s}{\PYZdq{}} \PY{o}{\PYZlt{}}\PY{o}{\PYZlt{}} \PY{n}{rect\PYZus{}length} \PY{o}{\PYZlt{}}\PY{o}{\PYZlt{}} \PY{l+s}{\PYZdq{}}\PY{l+s}{ and width = }\PY{l+s}{\PYZdq{}} \PY{o}{\PYZlt{}}\PY{o}{\PYZlt{}} \PY{n}{rect\PYZus{}width}\PY{p}{;}
\end{Verbatim}
\end{tcolorbox}

    \begin{Verbatim}[commandchars=\\\{\}]
Rectangle's length = 11.2 and width = 6.6
    \end{Verbatim}

    \begin{tcolorbox}[breakable, size=fbox, boxrule=1pt, pad at break*=1mm,colback=cellbackground, colframe=cellborder]
\prompt{In}{incolor}{8}{\boxspacing}
\begin{Verbatim}[commandchars=\\\{\}]
\PY{c+c1}{// step 2 and 3: calculate and display the area and perimeter}
\PY{n}{cout} \PY{o}{\PYZlt{}}\PY{o}{\PYZlt{}} \PY{l+s}{\PYZdq{}}\PY{l+s}{area of the rectangle: }\PY{l+s}{\PYZdq{}} \PY{o}{\PYZlt{}}\PY{o}{\PYZlt{}} \PY{n}{rect\PYZus{}length} \PY{o}{*} \PY{n}{rect\PYZus{}width} \PY{o}{\PYZlt{}}\PY{o}{\PYZlt{}} \PY{n}{endl}\PY{p}{;}
\PY{n}{cout} \PY{o}{\PYZlt{}}\PY{o}{\PYZlt{}} \PY{l+s}{\PYZdq{}}\PY{l+s}{perimeter of the rectangle: }\PY{l+s}{\PYZdq{}} \PY{o}{\PYZlt{}}\PY{o}{\PYZlt{}} \PY{l+m+mi}{2}\PY{o}{*}\PY{p}{(}\PY{n}{rect\PYZus{}length}\PY{o}{+}\PY{n}{rect\PYZus{}width}\PY{p}{)} \PY{o}{\PYZlt{}}\PY{o}{\PYZlt{}} \PY{n}{endl}\PY{p}{;}
\end{Verbatim}
\end{tcolorbox}

    \begin{Verbatim}[commandchars=\\\{\}]
area of the rectangle: 73.92
perimeter of the rectangle: 35.6
    \end{Verbatim}

    \hypertarget{demo-programs}{%
\subsubsection{demo programs}\label{demo-programs}}

\begin{itemize}
\tightlist
\item
  see the complete program here \url{demos/stdio/rectangle/main.cpp}
\end{itemize}

    \hypertarget{example-2-convert-decimal-number-to-binary}{%
\subsubsection{example 2: convert decimal number to
binary}\label{example-2-convert-decimal-number-to-binary}}

\begin{itemize}
\tightlist
\item
  let's convert \((13)_{10}\) to binary \((?)_2\)?

  \begin{itemize}
  \tightlist
  \item
    from manual calculation in Chapter 02, we know: \((13)_{10}\)
    -\textgreater{} \((1101)_2\)
  \end{itemize}
\item
  let's use algorithm defined in Chapter 02:

  \begin{enumerate}
  \def\labelenumi{\arabic{enumi}.}
  \tightlist
  \item
    repeteadly divide the decimal number by base 2 until the quotient
    becomes 0
  \item
    collect the remainders in reverse order

    \begin{itemize}
    \tightlist
    \item
      the first remainder becomes the last bit (least significant) in
      binary
    \end{itemize}
  \end{enumerate}
\item
  let's try to convert the above algorithm into C++ code
\end{itemize}

    \begin{tcolorbox}[breakable, size=fbox, boxrule=1pt, pad at break*=1mm,colback=cellbackground, colframe=cellborder]
\prompt{In}{incolor}{1}{\boxspacing}
\begin{Verbatim}[commandchars=\\\{\}]
\PY{c+cp}{\PYZsh{}}\PY{c+cp}{include} \PY{c+cpf}{\PYZlt{}iostream\PYZgt{}}\PY{c+c1}{ // cin, cout}
\PY{c+cp}{\PYZsh{}}\PY{c+cp}{include} \PY{c+cpf}{\PYZlt{}string\PYZgt{}}\PY{c+c1}{ // basic\PYZus{}string, to\PYZus{}string}

\PY{k}{using} \PY{k}{namespace} \PY{n+nn}{std}\PY{p}{;} \PY{c+c1}{// std::cin, std::cout, std::endl, etc.}
\end{Verbatim}
\end{tcolorbox}

    \begin{tcolorbox}[breakable, size=fbox, boxrule=1pt, pad at break*=1mm,colback=cellbackground, colframe=cellborder]
\prompt{In}{incolor}{2}{\boxspacing}
\begin{Verbatim}[commandchars=\\\{\}]
\PY{c+c1}{// decimal to binary conversion requires to calculate both quotient and remainder}
\PY{k}{const} \PY{k+kt}{int} \PY{n}{divisor} \PY{o}{=} \PY{l+m+mi}{2}\PY{p}{;} \PY{c+c1}{// divisor is contant name whose value can\PYZsq{}t be changed once initialized}
\PY{k+kt}{int} \PY{n}{dividend}\PY{p}{;}
\PY{k+kt}{int} \PY{n}{quotient}\PY{p}{,} \PY{n}{remain}\PY{p}{;}
\PY{n}{string} \PY{n}{answer}\PY{p}{;} \PY{c+c1}{// collect remainders by prepending as a string}
\end{Verbatim}
\end{tcolorbox}

    \begin{tcolorbox}[breakable, size=fbox, boxrule=1pt, pad at break*=1mm,colback=cellbackground, colframe=cellborder]
\prompt{In}{incolor}{3}{\boxspacing}
\begin{Verbatim}[commandchars=\\\{\}]
\PY{n}{answer} \PY{o}{=} \PY{l+s}{\PYZdq{}}\PY{l+s}{\PYZdq{}}\PY{p}{;} \PY{c+c1}{// variable to collect the binary answer}
\PY{n}{quotient} \PY{o}{=} \PY{l+m+mi}{13}\PY{p}{;} \PY{c+c1}{//start with the decimal 13}
\end{Verbatim}
\end{tcolorbox}

    \begin{tcolorbox}[breakable, size=fbox, boxrule=1pt, pad at break*=1mm,colback=cellbackground, colframe=cellborder]
\prompt{In}{incolor}{4}{\boxspacing}
\begin{Verbatim}[commandchars=\\\{\}]
\PY{c+c1}{// copy the quotient into dividend to divide it}
\PY{n}{dividend} \PY{o}{=} \PY{n}{quotient}\PY{p}{;}
\PY{n}{remain} \PY{o}{=} \PY{n}{dividend}\PY{o}{\PYZpc{}}\PY{n}{divisor}\PY{p}{;} \PY{c+c1}{// find the remainder}
\PY{n}{quotient} \PY{o}{=} \PY{n}{dividend}\PY{o}{/}\PY{n}{divisor}\PY{p}{;} \PY{c+c1}{// find the quotient}
\PY{c+c1}{// print intermediate results; help us see and plan further computation}
\PY{n}{cout} \PY{o}{\PYZlt{}}\PY{o}{\PYZlt{}} \PY{n}{dividend} \PY{o}{\PYZlt{}}\PY{o}{\PYZlt{}} \PY{l+s+sc}{\PYZsq{}}\PY{l+s+sc}{/}\PY{l+s+sc}{\PYZsq{}} \PY{o}{\PYZlt{}}\PY{o}{\PYZlt{}} \PY{n}{divisor} \PY{o}{\PYZlt{}}\PY{o}{\PYZlt{}} \PY{l+s}{\PYZdq{}}\PY{l+s}{ =\PYZgt{} quotient: }\PY{l+s}{\PYZdq{}} \PY{o}{\PYZlt{}}\PY{o}{\PYZlt{}} \PY{n}{quotient} \PY{o}{\PYZlt{}}\PY{o}{\PYZlt{}} \PY{l+s}{\PYZdq{}}\PY{l+s}{ remainder: }\PY{l+s}{\PYZdq{}} \PY{o}{\PYZlt{}}\PY{o}{\PYZlt{}} \PY{n}{remain} \PY{o}{\PYZlt{}}\PY{o}{\PYZlt{}} \PY{n}{endl}\PY{p}{;}
\PY{n}{answer} \PY{o}{=} \PY{n}{to\PYZus{}string}\PY{p}{(}\PY{n}{remain}\PY{p}{)} \PY{o}{+} \PY{n}{answer}\PY{p}{;} \PY{c+c1}{// prepend remainder to answer}
\PY{c+c1}{// is quotient 0?}
\end{Verbatim}
\end{tcolorbox}

    \begin{Verbatim}[commandchars=\\\{\}]
13/2 => quotient: 6 remainder: 1
    \end{Verbatim}

            \begin{tcolorbox}[breakable, size=fbox, boxrule=.5pt, pad at break*=1mm, opacityfill=0]
\prompt{Out}{outcolor}{4}{\boxspacing}
\begin{Verbatim}[commandchars=\\\{\}]
"1"
\end{Verbatim}
\end{tcolorbox}
        
    \begin{tcolorbox}[breakable, size=fbox, boxrule=1pt, pad at break*=1mm,colback=cellbackground, colframe=cellborder]
\prompt{In}{incolor}{5}{\boxspacing}
\begin{Verbatim}[commandchars=\\\{\}]
\PY{c+c1}{// further divide quotient}
\PY{n}{dividend} \PY{o}{=} \PY{n}{quotient}\PY{p}{;}
\PY{n}{remain} \PY{o}{=} \PY{n}{dividend}\PY{o}{\PYZpc{}}\PY{n}{divisor}\PY{p}{;}
\PY{n}{quotient} \PY{o}{=} \PY{n}{dividend}\PY{o}{/}\PY{n}{divisor}\PY{p}{;}
\PY{c+c1}{// print intermediate results; help us see and plan further computation}
\PY{n}{cout} \PY{o}{\PYZlt{}}\PY{o}{\PYZlt{}} \PY{n}{dividend} \PY{o}{\PYZlt{}}\PY{o}{\PYZlt{}} \PY{l+s+sc}{\PYZsq{}}\PY{l+s+sc}{/}\PY{l+s+sc}{\PYZsq{}} \PY{o}{\PYZlt{}}\PY{o}{\PYZlt{}} \PY{n}{divisor} \PY{o}{\PYZlt{}}\PY{o}{\PYZlt{}} \PY{l+s}{\PYZdq{}}\PY{l+s}{ =\PYZgt{} quotient: }\PY{l+s}{\PYZdq{}} \PY{o}{\PYZlt{}}\PY{o}{\PYZlt{}} \PY{n}{quotient} \PY{o}{\PYZlt{}}\PY{o}{\PYZlt{}} \PY{l+s}{\PYZdq{}}\PY{l+s}{ remainder: }\PY{l+s}{\PYZdq{}} \PY{o}{\PYZlt{}}\PY{o}{\PYZlt{}} \PY{n}{remain} \PY{o}{\PYZlt{}}\PY{o}{\PYZlt{}} \PY{n}{endl}\PY{p}{;}
\PY{n}{answer} \PY{o}{=} \PY{n}{to\PYZus{}string}\PY{p}{(}\PY{n}{remain}\PY{p}{)} \PY{o}{+} \PY{n}{answer}\PY{p}{;} \PY{c+c1}{// prepend remainder to answer}
\PY{c+c1}{// is quotient 0?}
\end{Verbatim}
\end{tcolorbox}

    \begin{Verbatim}[commandchars=\\\{\}]
6/2 => quotient: 3 remainder: 0
    \end{Verbatim}

            \begin{tcolorbox}[breakable, size=fbox, boxrule=.5pt, pad at break*=1mm, opacityfill=0]
\prompt{Out}{outcolor}{5}{\boxspacing}
\begin{Verbatim}[commandchars=\\\{\}]
"01"
\end{Verbatim}
\end{tcolorbox}
        
    \begin{tcolorbox}[breakable, size=fbox, boxrule=1pt, pad at break*=1mm,colback=cellbackground, colframe=cellborder]
\prompt{In}{incolor}{6}{\boxspacing}
\begin{Verbatim}[commandchars=\\\{\}]
\PY{c+c1}{// further divide quotient}
\PY{n}{dividend} \PY{o}{=} \PY{n}{quotient}\PY{p}{;}
\PY{n}{remain} \PY{o}{=} \PY{n}{dividend}\PY{o}{\PYZpc{}}\PY{n}{divisor}\PY{p}{;}
\PY{n}{quotient} \PY{o}{=} \PY{n}{dividend}\PY{o}{/}\PY{n}{divisor}\PY{p}{;}
\PY{c+c1}{// print intermediate results; help us see and plan further computation}
\PY{n}{cout} \PY{o}{\PYZlt{}}\PY{o}{\PYZlt{}} \PY{n}{dividend} \PY{o}{\PYZlt{}}\PY{o}{\PYZlt{}} \PY{l+s+sc}{\PYZsq{}}\PY{l+s+sc}{/}\PY{l+s+sc}{\PYZsq{}} \PY{o}{\PYZlt{}}\PY{o}{\PYZlt{}} \PY{n}{divisor} \PY{o}{\PYZlt{}}\PY{o}{\PYZlt{}} \PY{l+s}{\PYZdq{}}\PY{l+s}{ =\PYZgt{} quotient: }\PY{l+s}{\PYZdq{}} \PY{o}{\PYZlt{}}\PY{o}{\PYZlt{}} \PY{n}{quotient} \PY{o}{\PYZlt{}}\PY{o}{\PYZlt{}} \PY{l+s}{\PYZdq{}}\PY{l+s}{ remainder: }\PY{l+s}{\PYZdq{}} \PY{o}{\PYZlt{}}\PY{o}{\PYZlt{}} \PY{n}{remain} \PY{o}{\PYZlt{}}\PY{o}{\PYZlt{}} \PY{n}{endl}\PY{p}{;}
\PY{n}{answer} \PY{o}{=} \PY{n}{to\PYZus{}string}\PY{p}{(}\PY{n}{remain}\PY{p}{)} \PY{o}{+} \PY{n}{answer}\PY{p}{;} \PY{c+c1}{// prepend remainder to answer}
\PY{c+c1}{// is quotient 0?}
\end{Verbatim}
\end{tcolorbox}

    \begin{Verbatim}[commandchars=\\\{\}]
3/2 => quotient: 1 remainder: 1
    \end{Verbatim}

            \begin{tcolorbox}[breakable, size=fbox, boxrule=.5pt, pad at break*=1mm, opacityfill=0]
\prompt{Out}{outcolor}{6}{\boxspacing}
\begin{Verbatim}[commandchars=\\\{\}]
"101"
\end{Verbatim}
\end{tcolorbox}
        
    \begin{tcolorbox}[breakable, size=fbox, boxrule=1pt, pad at break*=1mm,colback=cellbackground, colframe=cellborder]
\prompt{In}{incolor}{7}{\boxspacing}
\begin{Verbatim}[commandchars=\\\{\}]
\PY{c+c1}{// further divide quotient}
\PY{n}{dividend} \PY{o}{=} \PY{n}{quotient}\PY{p}{;}
\PY{n}{remain} \PY{o}{=} \PY{n}{dividend}\PY{o}{\PYZpc{}}\PY{n}{divisor}\PY{p}{;}
\PY{n}{quotient} \PY{o}{=} \PY{n}{dividend}\PY{o}{/}\PY{n}{divisor}\PY{p}{;}
\PY{c+c1}{// print intermediate results; help us see and plan further computation}
\PY{n}{cout} \PY{o}{\PYZlt{}}\PY{o}{\PYZlt{}} \PY{n}{dividend} \PY{o}{\PYZlt{}}\PY{o}{\PYZlt{}} \PY{l+s+sc}{\PYZsq{}}\PY{l+s+sc}{/}\PY{l+s+sc}{\PYZsq{}} \PY{o}{\PYZlt{}}\PY{o}{\PYZlt{}} \PY{n}{divisor} \PY{o}{\PYZlt{}}\PY{o}{\PYZlt{}} \PY{l+s}{\PYZdq{}}\PY{l+s}{ =\PYZgt{} quotient: }\PY{l+s}{\PYZdq{}} \PY{o}{\PYZlt{}}\PY{o}{\PYZlt{}} \PY{n}{quotient} \PY{o}{\PYZlt{}}\PY{o}{\PYZlt{}} \PY{l+s}{\PYZdq{}}\PY{l+s}{ remainder: }\PY{l+s}{\PYZdq{}} \PY{o}{\PYZlt{}}\PY{o}{\PYZlt{}} \PY{n}{remain} \PY{o}{\PYZlt{}}\PY{o}{\PYZlt{}} \PY{n}{endl}\PY{p}{;}
\PY{n}{answer} \PY{o}{=} \PY{n}{to\PYZus{}string}\PY{p}{(}\PY{n}{remain}\PY{p}{)} \PY{o}{+} \PY{n}{answer}\PY{p}{;} \PY{c+c1}{// prepend remainder to answer}
\PY{c+c1}{// is quotient 0?}
\end{Verbatim}
\end{tcolorbox}

    \begin{Verbatim}[commandchars=\\\{\}]
1/2 => quotient: 0 remainder: 1
    \end{Verbatim}

            \begin{tcolorbox}[breakable, size=fbox, boxrule=.5pt, pad at break*=1mm, opacityfill=0]
\prompt{Out}{outcolor}{7}{\boxspacing}
\begin{Verbatim}[commandchars=\\\{\}]
"1101"
\end{Verbatim}
\end{tcolorbox}
        
    \begin{tcolorbox}[breakable, size=fbox, boxrule=1pt, pad at break*=1mm,colback=cellbackground, colframe=cellborder]
\prompt{In}{incolor}{9}{\boxspacing}
\begin{Verbatim}[commandchars=\\\{\}]
\PY{c+c1}{// stop division; display the answer}
\PY{n}{cout} \PY{o}{\PYZlt{}}\PY{o}{\PYZlt{}} \PY{l+s}{\PYZdq{}}\PY{l+s}{13  decimal = }\PY{l+s}{\PYZdq{}} \PY{o}{\PYZlt{}}\PY{o}{\PYZlt{}} \PY{n}{answer} \PY{o}{\PYZlt{}}\PY{o}{\PYZlt{}} \PY{l+s}{\PYZdq{}}\PY{l+s}{ binary }\PY{l+s}{\PYZdq{}} \PY{o}{\PYZlt{}}\PY{o}{\PYZlt{}} \PY{n}{endl}\PY{p}{;}
\end{Verbatim}
\end{tcolorbox}

    \begin{Verbatim}[commandchars=\\\{\}]
13  decimal = 1101 binary
    \end{Verbatim}

    \hypertarget{above-code-as-a-complete-c-program}{%
\subsubsection{Above code as a complete C++
program}\label{above-code-as-a-complete-c-program}}

\begin{itemize}
\tightlist
\item
  see \url{demos/stdio/decToBin/main.cpp}
\end{itemize}

\hypertarget{a-generic-c-program-to-convert-any-decimal-to-binary}{%
\subsubsection{A generic C++ program to convert any decimal to
binary}\label{a-generic-c-program-to-convert-any-decimal-to-binary}}

\begin{itemize}
\tightlist
\item
  basic building blocks covered so far is able to find the solution
  using Jupyter notebook

  \begin{itemize}
  \tightlist
  \item
    however, we've not learned enough to write a generic program that
    can convert any integer into binary, just yet!
  \end{itemize}
\item
  we'll revisit this problem as we learn more concepts, such as
  conditional statements and loops
\end{itemize}

    \hypertarget{labs}{%
\subsection{Labs}\label{labs}}

\begin{enumerate}
\def\labelenumi{\arabic{enumi}.}
\tightlist
\item
  Standard IO Lab

  \begin{itemize}
  \tightlist
  \item
    write a C++ program that produces the following output on console
  \item
    use the partial solution provided in \url{labs/stdio/main.cpp}
  \item
    observe and note how the special symbols such as single quote,
    double quotes and black slashes
  \item
    run the program as it is using the provided make file in the stdio
    folder
  \item
    complete the rest of the ASCII Art by fixing all the FIXMEs
  \item
    write \#FIXED next to each FIXME
  \end{itemize}

\begin{verbatim}
    |\_/|       *******************************     (\_/)
   / @ @ \      *      ASCII Art              *    (='.'=)
  ( > 0 < )     *      Author: <Your Name>    *  ( " )_( " )
    >>x<<       *      CS Foundation Course   *
    / O \       *******************************
\end{verbatim}
\end{enumerate}

    \hypertarget{exercises}{%
\subsection{Exercises}\label{exercises}}

\begin{enumerate}
\def\labelenumi{\arabic{enumi}.}
\tightlist
\item
  Write a C++ program including algorithm steps that calculates area and
  perimeter of a circle.
\item
  Write a C++ program including algorithm steps that calculates Body
  Mass Index (BMI) of a person.

  \begin{itemize}
  \tightlist
  \item
    More information on BMI -
    https://www.nhlbi.nih.gov/health/educational/lose\_wt/BMI/bmicalc.htm
  \item
    Formula
    \href{https://www.cdc.gov/healthyweight/assessing/bmi/childrens_bmi/childrens_bmi_formula.html\#:~:text=The\%20formula\%20for\%20BMI\%20is,to\%20convert\%20this\%20to\%20meters.\&text=When\%20using\%20English\%20measurements\%2C\%20pounds\%20should\%20be\%20divided\%20by\%20inches\%20squared}{here}.
  \item
    a sample solution is provided here
    \url{exercises/stdio/BMI/main.cpp}
  \end{itemize}
\item
  Write a C++ program including algorithm steps that calculates area and
  perimeter of a triangle given three sides.

  \begin{itemize}
  \tightlist
  \item
    Hint: use Heron's formula to find area with three sides.
  \end{itemize}
\item
  Write a C++ program that converts hours into seconds.

  \begin{itemize}
  \tightlist
  \item
    e.g.~given 2 hours, program should print 7200 as answer.
  \end{itemize}
\item
  Write a C++ program that converts seconds into hours, minutes and
  seconds.

  \begin{itemize}
  \tightlist
  \item
    e.g.~given 3600 seconds, program should print 1 hour, 0 minute and 0
    second.
  \item
    e.g.~given 3661 seconds, program should print 1 hour, 1 minute and 1
    second.
  \item
    Hint: use series of division and module operators
  \end{itemize}
\item
  Convert your full name into binary code using Jupyter Notebook.
\end{enumerate}

\hypertarget{kattis-problems}{%
\subsection{Kattis Problems}\label{kattis-problems}}

\begin{enumerate}
\def\labelenumi{\arabic{enumi}.}
\tightlist
\item
  Solving for Carrots - https://open.kattis.com/problems/carrots

  \begin{itemize}
  \tightlist
  \item
    a simple standard input/output problem; just print the second number
    in first line
  \item
    Hint: simply print P
  \item
    see sample solution in \url{demo_programs/Ch03/carrots} folder
  \end{itemize}
\item
  R2 - https://open.kattis.com/problems/r2

  \begin{itemize}
  \tightlist
  \item
    Hint: simply output \(2*S-R1\)
  \end{itemize}
\item
  Spavanac - https://open.kattis.com/problems/spavanac

  \begin{itemize}
  \tightlist
  \item
    Hint: convert min+hour to minute; subtract 45 and convert the result
    back to hour and minute and print them
  \end{itemize}
\item
  Add Two Numbers - https://open.kattis.com/problems/addtwonumbers

  \begin{itemize}
  \tightlist
  \item
    Hint: read two numbers and print their sum
  \end{itemize}
\item
  Echo Echo Echo - https://open.kattis.com/problems/echoechoecho

  \begin{itemize}
  \tightlist
  \item
    Hint: read the word; print the word three times
  \end{itemize}
\end{enumerate}

    \hypertarget{testing-kattis-provided-samples}{%
\subsection{Testing Kattis provided
samples}\label{testing-kattis-provided-samples}}

\begin{itemize}
\tightlist
\item
  one way to check for the sample input and output is by manually typing
  the input and comparing the results

  \begin{itemize}
  \tightlist
  \item
    input can be long and output can be tedious to compare
  \item
    Kattis expects output to be 100\% accuracte to the space
  \end{itemize}
\end{itemize}

\hypertarget{recommended-way-to-automate-the-process-to-solve-kattis-problems}{%
\subsubsection{Recommended way to automate the process to solve Kattis
problems}\label{recommended-way-to-automate-the-process-to-solve-kattis-problems}}

\begin{itemize}
\tightlist
\item
  download the samples provided in a compressed \texttt{.zip} file
\item
  unzip the file; it'll create a folder with the same name as the
  problem name or zip file name
\item
  create a \texttt{problemName.cpp} solution file inside the same folder
  where the sample files are
\item
  then do the following steps:

  \begin{itemize}
  \item
    open a terminal on Mac/Linux/WSL
  \item
    change working directory to a problem folder, e.g.~carrots

\begin{Shaded}
\begin{Highlighting}[]
\BuiltInTok{pwd}
\BuiltInTok{cd} \OperatorTok{\textless{}}\NormalTok{path to carrots folder}\OperatorTok{\textgreater{}}
\FunctionTok{ls}
\end{Highlighting}
\end{Shaded}
  \item
    directly compile using g++ or create and use a Makefile

\begin{Shaded}
\begin{Highlighting}[]
\ExtensionTok{g++} \AttributeTok{{-}std}\OperatorTok{=}\NormalTok{c++17 carrots.cpp}
\end{Highlighting}
\end{Shaded}
  \item
    run Kattis provided sample test cases e.g.~if 1.in and 1.ans are
    corresponding sample test files
  \item
    read the sample 1.in and pipe it to \texttt{a.out} program and pipe
    the answer from the program to diff to compare against 1.ans

\begin{Shaded}
\begin{Highlighting}[]
\FunctionTok{cat}\NormalTok{ 1.in }\KeywordTok{|} \ExtensionTok{./a.out} \KeywordTok{|} \FunctionTok{diff} \AttributeTok{{-}}\NormalTok{ 1.ans}
\FunctionTok{cat}\NormalTok{ 2.in }\KeywordTok{|} \ExtensionTok{./a.out} \KeywordTok{|} \FunctionTok{diff} \AttributeTok{{-}}\NormalTok{ 2.ans}
\end{Highlighting}
\end{Shaded}
  \item
    if the program's answer is correct, you'll not see any difference or
    output on the terminal
  \end{itemize}
\item
  once your program provides correct result as shown in the
  corresponding output, upload your \texttt{.cpp} source file to the
  Kattis to test against all the hidden test samples

  \begin{itemize}
  \tightlist
  \item
    Kattis will compile and execute your program to test against the
    other samples
  \end{itemize}
\item
  Kattis will either accept your solution or reject with some simple
  feedback such as wrong answer
\end{itemize}

    \hypertarget{summary}{%
\subsection{Summary}\label{summary}}

\begin{itemize}
\tightlist
\item
  this chapter covered reading data from common input stream (standard
  input)
\item
  this chapter covered writing data to common output stream (standard
  output)
\item
  covered escape character, sequences and their usage
\item
  we also learned about composing more meaningful programs with two
  examples
\item
  exercises and problems with sample solutions
\end{itemize}

    \begin{tcolorbox}[breakable, size=fbox, boxrule=1pt, pad at break*=1mm,colback=cellbackground, colframe=cellborder]
\prompt{In}{incolor}{ }{\boxspacing}
\begin{Verbatim}[commandchars=\\\{\}]

\end{Verbatim}
\end{tcolorbox}


    % Add a bibliography block to the postdoc
    
    
    
\end{document}
